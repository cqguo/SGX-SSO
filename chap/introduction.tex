\section{Introduction}
\label{sec:intro}

%SSO的特点
%SSO的现状
Single sign-on (SSO) systems, such as OAuth, OpenID Connect and SAML, have been widely adopted nowadays as a convenient web authentication mechanism. SSO delegates user authentication on websites to a third party, so-called identity providers (IdPs), so that users can access different services on cooperating sites, so-called relying parties (RPs), via a single authentication process. Using SSO, a user no longer needs to maintain multiple credentials for different RPs, instead, she only maintains the credential for the IdP, which further provides identity proofs to RPs.
For RPs, SSO shifts the burden of user authentication to IdPs and therefore reduces their own security risks and costs. As a result, SSO has been widely integrated with modern web systems.
According to Alexa~\cite{Alexa}, 80 websites among the top-100 websites integrate SSO services. Meanwhile, many email and social networking providers such as Google, Facebook, Twitter, etc. have been actively serving as social identity providers to support social login.

However, the wide adoption of SSO also raises new privacy concerns. 
NIST~\cite{NIST2017draft} indicates that 
%Currently, more and more companies are interested in the users' online trace for business practices, such as accurate delivery advertising. For example, large internet service providers, such as Google and Facebook, are interested in collecting users' online behavioral information for various purposes (e.g., Screenwise Meter~\cite{googlenews} and Onavo~\cite{Onavo}). However, by simply serving the IdP role, these companies can easily collect a large amount of data to reconstruct users' online traces. 
curious IdP or multiple collusive RPs could break the users' privacy as follows.
\begin{itemize}
\item {\em IdP-based login tracing}. The IdP knows the identities of the RP and user in each single login instance, to generate the identity proof.
As a result, a curious IdP could discover all the RPs that the victim user attempts to visit and profile her online activities.
\item {\em RP-based identity linkage}. The RP learns a user's identity from the identify proof.
When the IdP generates identity proofs for a user, if the same user identifier is used in identity proofs generated for different RPs, malicious RPs could collude to not only link the user's login activities at different RPs for online tracking but also associate her attributes across multiple RPs.
\end{itemize}

The IdP-based login tracing and RP-based identity linkage would bring the real threat in reality. Imagine that, a user concerning her privacy would avoid to leave her full sensitive information at an application. She only leaves parts of her sensitive messages at each applications, for example, using real name on a social website and address on shopping website. And she would try not to leave any linkable message to avoid applications combining her informations, such as email.
% for example, if she leave the email on each applications, they can combine the parts of informations through the email. 
However, the privacy leaks in SSO systems make her effort in vain. As long as a user employs the SSO system, such as Google Account, to log in to these applications, the applications providers and Google can combine her information based on the SSO account.


To protect user privacy, the Pairwise Pseudonymous Identifier (PPID) is suggested by NIST~\cite{NIST2017draft} and adopted by some popular SSO protocols, such as OpenID Connect~\cite{OpenIDConnect,} and SAML~\cite{SAMLIdentifier}. 
In the PPID SSO system, IdP offers a user multiple individual IDs for different RPs. 
Therefore, an RP cannot  correlate the received  PPID  with the user's PPID at another RP. Thus, collusive RPs cannot link a user's logins from her PPIDs. 
Recently more and more IdPs have adopted PPID to protect user privacy. Moreover, Active Directory Federation Services and Oracle Access Management support the use of PPID, the privacy protection scheme suggested in OIDC protocol~\cite{MS, Oracle}. Identity service providers such as NORDIC APIS and CURITY suggest adopting PPID in SSO to protect user privacy~\cite{Nordic, Curity}.

There have already been many works on protecting user privacy in SSO systems. 
However, to the best of our knowledge, none of the existing techniques can be integrated into the existing PPID systems. 
Here we give a brief introduction of existing solutions for privacy-preserving SSO and explain the flaws of these schemes. 
\begin{itemize}
\item {\em Simply hiding RP ID from IdP. }For example, SPRESSO~\cite{SPRESSO} were proposed to defend against IdP-based login tracing by hiding RP ID from IdP. It uses the encrypted RP ID instead, and IdP issues the identity proof for this one-time ID. However, PPID is not available in this type of solution, as IdP cannot offer an RP the constant user ID with one-time irrelevant RP ID.
\item {\em Proving user identity based on zero-knowledge proof. }In this type of solution , such as EL PASSO~\cite{ZhangKSZR21} and UnlimitID~\cite{IsaakidisHD16}, the user needs to keep a secret $s$ and requires IdP to generate an identity proof for blinded $s$. Then user has to prove that she is the owner of $s$ to RP without exposing $s$ to RP. However, it is not convenient for user login on multiple devices. The user's identity is associated with the private $s$, therefore, if the user wants to log in to the RP on a new device, she must import the large $s$ into this device. For security consideration, the $s$ must be too long for user to remember. 
\item {\em Completely anonymous SSO system. }Anonymous SSO scheme is proposed to hide the user's identity to both the IdP and RPs with many methods. For example, in the anonymous SSO ~\cite{HanCSTW18}, it allows a user to visit the RP without exposing her identity to both IdP and RP based on zero-knowledge proof. However, it can only be applied to the anonymous services that do not identify the user, but not available in current personalized internet service.
\end{itemize}


%As discussed above, none of the existing SSO systems defend against both IdP-based login tracing and RP-based identity linkage, and provide the convenient SSO service on multiple devices at the same time.
%We remark that the problems cannot be solved, such as simply combining existing solutions together. The challenge is that, there is not a simple way for IdP to provide PPID without knowing the RP's identity.

In this paper, we propose Enhancing the User Privacy of SSO by Integrating PPID and SGX (named UP-SSO), the enhanced PPID SSO system with comprehensive protection against bosh IdP-based login tracing and RP-based identity linkage.
The key idea of UP-SSO is shifting part of IdP service, PPID generation, from server to user client. 
%Therefore, IdP server can issue the identity token containing PPID, only requiring an one-time transformed RP ID (e.g., encrypted RP ID). 
The user-side IdP service takes the responsibility for transforming RP ID (as an identity token must contain the RP ID to avoid misuse of token~\cite{YangLCZ18, WangZLG16, MainkaMS16, MainkaMSW17, YangLCZ18} ) and generating PPID. 
For IdP, it can only achieve a encrypted one-time RP ID, so that the IdP-based login tracing is not impossible.
To be noticed is that the user-side service must be deployed at the user controlled and IdP trusted environment, i.e., Intel SGX. With SGX, the secure hardware supported by intel CPU, the user-side service can be deployed at user's PC, and it can be verified by IdP through $remote attestation$, the function provided by SGX.
Moreover, other essential IdP services, such as user authentication and identity proof issuing,  must be completed at server side for security consideration. 

The overview of UP-SSO is that 


We summarize our contributions as follows.
%我们的贡献
%提出协议
%考虑能否根据模型进行分析
%实现原型系统
%The main contributions of UPPRESSO are as follows:
\begin{itemize}
\item We propose the comprehensive solution to hide the users' login traces from both the curious IdP and malicious collusive RPs for convenient SSO system.
\item We formally analyze the security of XXX and show that it guarantees the security, while the users' login traces are well protected.
\item We have implemented a prototype of XXX, and compare the performance of the XXX prototype with the state-of-the-art SSO systems (e.g., OIDC), and demonstrate its efficiency.
\end{itemize}

%文章结构
The rest of the paper is organized as follows. We first introduce the background in Section~\ref{sec:background}. Then, we describe the threat model, and our XXX design in Sections~\ref{sec:threatmodel} and \ref{sec:design}, followed by a formal analysis of security and privacy in Section~\ref{sec:analysis}. We provide the implementation specifics and experiment evaluation in Section~\ref{sec:implementation}, discuss the related works in Section~\ref{sec:relatedwork}, and conclude our work in Section~\ref{sec:conclusion}.
