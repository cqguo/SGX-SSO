\newpage
\appendix
\section{Appendix: Web Model}
%\begin{appendices}
\subsection{Message Format}
Here we provide the details of the format of the messages we use to construct the ExtraF model.

\vspace{1mm}\noindent\textbf{HTTP Messages}.
An HTTP request message is the term of the form
\vspace{-3mm}
\begin{equation*}
\langle \mathtt{HTTPReq}, nonce, method, host, path, parameters, headers, body\rangle
\end{equation*}

An HTTP response message is the term of the form
\vspace{-2mm}
\begin{equation*}
    \myangle{\mathtt{HTTPResp}, nonce, status, headers, body}
\end{equation*}
 The details are defined as follows:
 \begin{itemize}
 \item \myss{\mathtt{HTTPReq}} and \myss{\mathtt{HTTPResp}} denote the types of messages.
 \item \myss{nonce} is a random number that maps the response to the corresponding request.
 \item \myss{method} is the HTTP methods, such as \myss{\mathtt{GET}} and \myss{\mathtt{POST}}.
 \item \myss{host} is the constant string domain of visited server.
 \item \myss{path} is the constant string representing the concrete resource of the server.
 \item \myss{parameters} contains the parameters carried by the url as the form \\ \myss{\myangle{\myangle{name, value}, \myangle{name, value}, \dotsc}}, for example, the \myss{parameters} in the url \myss{http://www.example.com?type=confirm}  is \myss{\myangle{\myangle{type, confirm}}}.
 \item \myss{headers} is the header content of each HTTP messages as the form \\ \myss{\myangle{\myangle{name, value}, \myangle{name, value}, \dotsc}}, such as \\ \myss{\langle\myangle{Referer, http://www.example.com},} \myss{\myangle{Cookies, c}\rangle}.
 \item \myss{body} is the body content carried by HTTP \myss{\mathtt{POST}} request or HTTP response in the form \myss{\myangle{\myangle{name, value}, \myangle{name, value}, \dotsc}}.
  \item \myss{status} is the HTTP status code defined by HTTP standard.
 \end{itemize}

\vspace{1mm}\noindent\textbf{URL}.
URL is a term \myss{\myangle{\mathtt{URL}, protocol,host,path,parameters}}, where \myss{\mathtt{URL}} is the type, \myss{protocol} is chosen in \myss{\{\mathtt{S}}, \myss{\mathtt{P}\}} as \myss{\mathtt{S}} stands for HTTPS and \myss{\mathtt{P}} stands for HTTP. The \myss{host, path}, and \myss{parameters} are the same as in HTTP messages.

\vspace{1mm}\noindent\textbf{Origin}.
An Origin is a term \myss{\myangle{host, protocol}} that stands for the specific domain used by the HTTP CORS policy, where \myss{host} and \myss{protocol} are the same as in URL.

\vspace{1mm}\noindent\textbf{POSTMESSAGE}.
PostMessage is used in the browser for transmitting messages between scripts from different origins. We define the postMessage as the form \myss{\myangle{\mathtt{POSTMESSAGE}, target, Content, Origin}}, where \myss{\mathtt{POSTMESSAGE}} is the type, \myss{target} is the constant nonce which stands for the receiver, \myss{Content} is the message transmitted and \myss{Origin} restricts the receiver's origin.

\vspace{1mm}\noindent\textbf{XMLHTTPREQUEST}.
XMLHTTPRequest is the HTTP message transmitted  by scripts in the browser. That is, the XMLHTTPRequest is converted from the HTTP message by the browser. The XMLHTTPRequest in the form \myss{\myangle{\mathtt{XMLHTTPREQUEST}, URL, methods, Body, nonce}} can be converted into HTTP request message by the browser, and \myss{\myangle{\mathtt{XMLHTTPREQUEST}, Body, nonce}}  is converted from HTTP response message.

\vspace{1mm}\noindent\textbf{ENCLAVEMESSAGE}.
EnclaveMessage is the message transmitted between Script Process and Enclave Application. It is defined as the term \\ \myss{\myangle{\mathtt{ENCLAVEMESSAGE}, Content}}.


\noindent\textbf{Data Operation}.
The data used in ExtraF are defined in the following forms:
\begin{itemize}
\item \textbf{Standardized Data} is the data in the fixed format, for instance, the HTTP request is the standardized data in the form \myss{\langle\mathtt{HTTPReq}}, \myss{nonce}, \myss{method}, \myss{host}, \myss{path}, \myss{parameters}, \myss{headers}, \myss{body\rangle}.  We assume there is an HTTP request \myss{r :=}  \myss{\langle\mathtt{HTTPReq}},  \myss{n},  \myss{\mathtt{GET}},  \myss{example.com},  \myss{/path},  \myss{\myangle{}},  \myss{\myangle{}},  \myss{\myangle{}\rangle}, here we define the operation on the $r$. That is, the elements in $r$ can be accessed in the form \myss{r.name}, such that \myss{r.method \equiv \mathtt{GET}},  \myss{r.path \equiv /path} and \myss{r.body \equiv \myangle{}}.
\item \textbf{Dictionary Data} is the data in the form \myss{\myangle{\myangle{name, value}, \myangle{name, value}, \dotsc}}, for instance the \myss{body} in HTTP request is dictionary data. We assume there is a \myss{body := \myangle{\myangle{username, alice}, \myangle{password, 123}}}, here we define the operation on the $body$. That is, we can access the elements in \myss{body} in the form \myss{body[name]}, such that \myss{body[username] \equiv alice} and \myss{body[password] \equiv 123}.
%We can also add the new attributes to the dictionary, for example after we set \myss{body[age] := 18}, the \myss{body} are changed into\myss{ \myangle{\myangle{username, alice}, \myangle{password, 123}, \myangle{age, 18}}}.
\end{itemize}

\subsection{Browser Model}
In UPPRESSO, we assume that the browsers are honest, therefore, we only need to analyze how the browsers interactive with the scripts. 
We first introduce the windows and documents of the browser model.

\vspace{1mm}
\noindent\textbf{Window}. A window \myss{w} is a term of the form \myss{w = \myangle{nonce, documents, opener}}, representing the  the concrete browser window in the system. The \myss{nonce} is the window reference to identify each windows. The \myss{documents} is the set of documents (defined below) including the current document and cached documents (for example, the documents can be viewed via the ``forward" and ``back" buttons in the browser). The \myss{opener} represents the window in which this window is created, for instance, while a user clicks the href in document \myss{d} and it creates a new window \myss{w}, there is \myss{w.opener \equiv d.nonce}.

\vspace{1mm}
\noindent\textbf{Document}. A document \myss{d} is a term of the form
\begin{multline*}
\langle nonce, location, referrer, script, scriptstate, \\
 scriptinputs, subwindows, active \rangle 
\end{multline*}
where document is the HTML content in the window.  The \myss{nonce} locates the document. \myss{Location} is the URL where the document is loaded. \myss{Referrer} is same as the Referer header defined in HTTP standard. The \myss{script} is the scripting process downloaded from each servers. \myss{scriptstate} is define by the script, different in each scripts. The \myss{scriptinputs} is the message transmitted into the scripting process. The \myss{subwindows} is the set of \myss{nonce} of document's created windows. \myss{active} represents whether this document is active or not.

A scripting process is the dependent process relying on the browser, which can be considered as a relation \myss{R} mapping a message input and a message output. And finally the browser will conduct the command in the output message. Here we give the description of the form of input and output.
\begin{itemize}
\setlength\itemsep{-2pt}
\item \textbf{Scripting Message Input. } The input is the term in the form
\begin{multline*}
\langle tree, docnonce, scriptstate, stateinputs,cookies,\\
localStorage, sessionStorage, ids, secret \rangle
\end{multline*}
\item \textbf{Scripting Message Output. }The output is the term in the form
\begin{equation*}
\langle scriptstate, cookies, localStorage,sessionStorage, command \rangle
\end{equation*}
\end{itemize}
The \myss{tree} is the relations of the opened windows and documents, which are visible to this script. \myss{Docnonce} is the document nonce. The  \myss{Scriptstate} is a term of the form defined by each script. \myss{Scriptinputs} is the message transmitted to script. However, the \myss{scriptinputs} is defined as standardized forms, for example, postMessage is one of the forms of \myss{scriptinputs}. \myss{Cookies} is the set of cookies that belong to the document's origin. \myss{LocalStorage} is the storage space for browser and \myss{sessionStorage} is the space for each HTTP sessions.  \myss{Ids} is the set of user IDs while \myss{secret} is the password to corresponding user ID. The \myss{command} is the operation which is to be conducted by the browser. Here we only introduce the form of commands used in XXX system. We have defined the postMessage and XMLHTTPRequest (for HTTP request) message which are the \myss{commands}. Moreover, a term in the form \myss{\myangle{\mathtt{IFRAME}, URL, WindowNonce}} asks the browser to create this document's subwindow and it visits the server with the URL.

\subsection{XXX Model}

\noindent\textbf{IdP server}.

\begin{breakablealgorithm}
  \caption{$R^i$}
  \label{alg1}
  \begin{algorithmic}[1]
  \Require{\myss{\myangle{a, f, m}, s}}
  \mystate{\myss{s':=s}}
  \mystate{\myss{n, method, path, parameters, headers, body} \textbf{such that}}\\
  \ \ \myss{\myangle{\mathtt{HTTPReq},n,method,path,parameters,headers,body} \equiv m}\\
  \ \ \textbf{if} \myss{possible}; \textbf{otherwise} stop \myss{\myangle{}, s'}
  \myif{path \equiv /script}
  \mystate{\myss{m':=\myangle{\mathtt{HTTPResp},n,200, \myangle{}, \mathtt{IdPScript}}}}
  \mystop{f, a, m'}
  
  \myelse{path \equiv /login}
  \mystate{\myss{cookie := headers[Cookie]}}
  \mystate{\myss{session := s'.sessions[cookie]}}
  \mystate{\myss{username:=body[username]}}
  \mystate{\myss{password:=body[password]}}
  \myif{password \not\equiv \mathtt{SecretOfID}(username)}
  \mystate{\myss{m' :=\myangle{\mathtt{HTTPResp},n,200,\myangle{},\mathtt{LoginFailure}}}}
  \mystop{f,a,m'}
  \EndIf
  \mystate{\myss{session[uid] := \mathtt{UIDOfUser}(username)}}
  \mystate{\myss{m' :=\myangle{\mathtt{HTTPResp},n,200,\myangle{},\mathtt{LoginSucess}}}}
  \mystop{f,a,m'}
  
  \myelse{path \equiv /requireUIDToken}
  \mystate{\myss{cookie := headers[Cookie]}}
  \mystate{\myss{session := s'.sessions[cookie]}}
  \mystate{\myss{uid := session[uid]}}
  \myif{uid \equiv \mathtt{null}}
  \mystate{\myss{m' := \myangle{\mathtt{HTTPResp},n,200,\myangle{},\mathtt{UnLogged}}}}
  \mystop{f,a,m'}
  \EndIf
  \mystate{\myss{token := \mathtt{GenerateToken}()}}
  \mystate{\myss{s'.Tokens := s'.Tokens + ^{\myangle{}}\myangle{uid, token}}}
  \mystate{\myss{m' := \myangle{\mathtt{HTTPResp},n,200,\myangle{},\myangle{\mathtt{Token}, token}}}}
  \mystop{f,a,m'}
  
  \myelse{path \equiv /requireUID}
  \mystate{\myss{UIDToken := body[UIDToken]}}
  \mystate{\myss{uid := \mathtt{FindUIDByToken(UIDToken)}}}
  \myif{uid \equiv null}
  \mystate{\myss{m' := \myangle{\mathtt{HTTPResp}, n, 200, \myangle{}, \mathtt{TokenError}}}}
  \mystop{f,a,m'}
  \EndIf
  \mystate{\myss{m' := \myangle{\mathtt{HTTPResp}, n, 200, \myangle{}, \myangle{\mathtt{UID}, uid}}}}
  \mystop{f,a,m'}
  
  \myelse{path \equiv /requireToken}
  \mystate{\myss{PRPID := body[PRPID]}}
  \mystate{\myss{PPID := body[PPID]}}
  \mystate{\myss{Content := \myangle{PRPID,PPID}}}
  \mystate{\myss{token := Content+Sign(Content,s'.SignKey)}}
  \mystate{\myss{m' := \myangle{\mathtt{HTTPResp}, n, 200, \myangle{}, \myangle{\mathtt{Token},token}}}}
  \mystop{f,a,m'}
  \EndIf
  \mystop{}
  \end{algorithmic}
\end{breakablealgorithm}

\noindent\textbf{RP server}.

\begin{breakablealgorithm}
  \caption{$R^r$}
  \label{alg2}
  \begin{algorithmic}[1]
  \Require {\myss{\myangle{a, f, m}, s}}
  \mystate{\myss{s':=s}}
  \mystate{\myss{n, method, path, parameters, headers, body} \textbf{such that}}\\
  \ \ \myss{\myangle{\mathtt{HTTPReq},n,method,path,parameters,headers,body} \equiv m}\\
  \ \ \textbf{if} \myss{possible}; \textbf{otherwise} stop \myss{\myangle{}, s'}
  \myif{path \equiv /script}
\mystate{\myss{m':=\myangle{\mathtt{HTTPResp},n,200, \myangle{}, \mathtt{RPScript}}}}
  \mystop{f, a, m'}
  
  \myelse{path \equiv /uploadToken}
  \mystate{\myss{cookie := headers[Cookie]}}
  \mystate{\myss{session := s'.sessions[cookie]}}
  \mystate{\myss{token := body[Token]}}
  \mystate{\myss{key := body[Key]}}
  \myif{\mathtt{CheckSig}(Token.Content, Token.Sig, s'.IdP.PubKey)}
  \mystate{\myss{PRPID := \mathtt{Encrypt}(s'.RPDomain,key)}}  
  \myif{PRPID \equiv token.Content.PRPID}
  \mystate{\myss{PPID := token.Content.PPID}}
  \myif{PPID \not \in \mathtt{ListOfUser}()}
  \mystate{\myss{\mathtt{RegisterUser}(PPID)}}
  \EndIf
  \mystate{\myss{session[user] := PPID}}
  \mystate{\myss{m' := \myangle{\mathtt{HTTPResp}, n, 200, \myangle{}, \mathtt{LoginSuccess}}}}
  \EndIf
  \EndIf
  \mystate{\myss{m' := \myangle{\mathtt{HTTPResp}, n, 200, \myangle{}, \mathtt{Fail}}}}
  \mystop{f, a, m'}
  \EndIf
  \mystop{}
  \end{algorithmic}
\end{breakablealgorithm}


\noindent\textbf{IdP script}.

\begin{breakablealgorithm}
  \caption{$script\_idp$}
  \label{alg3}
  \begin{algorithmic}[1]
  \Require{\myss{\langle tree, docnonce, scriptstate, scriptinputs, cookies, localStorage,}}
  \Statex {\myss{ sessionStorage, ids, secret \rangle}}  
  \mystate{\myss{ s' := scriptstate}}
  \mystate{\myss{command := \myangle{}}}
  \mystate{\myss{target := \mathtt{PARENTWINDOW}(tree,docnonce)}}
  \mystate{\myss{IdPDomain := s'.IdPDomain}}
  \SWITCH{\myss{s'.q}}
    \CASE{\myss{startLogin}}
    \mystate{\myss{username \in ids}}
    \mystate{\myss{Url := \myangle{\mathtt{URL}, \mathtt{S}, IdPDomain, /login, \myangle{}}}}
    \mystate{\myss{s'.refXHR :=  \mathtt{Random}()}}
    \mystate{\myss{command : = \langle \mathtt{XMLHTTPREQUEST}, Url, \mathtt{POST},}}  
    \Statex \myss{\ \ \ \ \ \ \ \ \ \ \myangle{\myangle{username, username}, \myangle{password, secret}}, s'.refXHR \rangle}
    \mystate{\myss{s'.q := expectLoginResult}}
    \ENDCASE
    
    \CASE{expectLoginResult}
      \mystate{\myss{pattern := \myangle{\mathtt{XMLHTTPREQUEST},Body,s'.refXHR}}}
      \mystate{\myss{input := \mathtt{CHOOSEINPUT}(scriptinputs,pattern) }}
      \myif{input \not\equiv \mathtt{null}}
      \myif{input.Body \not\equiv \mathtt{LoginSuccess}}
      \mystate{\myss{\textbf{stop}\ \myangle{}}}
      \EndIf
      \mystate{\myss{command := \myangle{\mathtt{POSTMESSAGE}, target, \mathtt{Ready}, \mathtt{null}}}}
      \mystate{\myss{s'.q := expectRPDomain}}
      \EndIf
      \ENDCASE
    
    \CASE{\myss{expectRPDomain}}
      \mystate{\myss{pattern := \myangle{\mathtt{POSTMESSAGE}, *, Content, *}}}
      \mystate{\myss{input := \mathtt{CHOOSEINPUT}(scriptinputs,pattern)}}
      \myif{input \not\equiv \mathtt{null}}
      \mystate{\myss{RPDomain := input.Content[RPDomain]}}
      \mystate{\myss{s'.Parameters[RPDomain] := RPDomain}}
      \mystate{\myss{Url := \myangle{\mathtt{URL}, \mathtt{S}, IdPDomain, /requireUIDToken,\myangle{} }}}
      \mystate{\myss{s'.refXHR :=  \mathtt{Random}()}}
      \mystate{\myss{command : = \langle\mathtt{XMLHTTPREQUEST}, Url, \mathtt{GET}, s'.refXHR\rangle}}
      \mystate{\myss{s'.q := expectUIDToken}}
       \EndIf
      \ENDCASE
      
    \CASE{expectUIDToken}
      \mystate{\myss{pattern := \myangle{\mathtt{XMLHTTPREQUEST},Body,s'.refXHR}}}
      \mystate{\myss{input := \mathtt{CHOOSEINPUT}(scriptinputs,pattern) }}
      \myif{input \not\equiv \mathtt{null}}
      \mystate{\myss{token := input.Body[UIDToken]}}
      \mystate{\myss{command := \myangle{\mathtt{ENCLAVEMESSAGE}, \myangle{\mathtt{UIDToken}, token}}}}
      \mystate{\myss{s'.q := expectIdentityToken}}
      \EndIf
      \ENDCASE
    
    \CASE{expectIdentityToken}
      \mystate{\myss{pattern := \myangle{\mathtt{ENCLAVEMESSAGE},Content}}}
      \mystate{\myss{input := \mathtt{CHOOSEINPUT}(scriptinputs,pattern) }}
      \myif{input \not\equiv \mathtt{null}}
      \mystate{\myss{token:=input.Content[Token]}}
      \mystate{\myss{key:=input.Content[Key]}}
      \mystate{\myss{command := \langle\mathtt{POSTMESSAGE}, target, \langle\myangle{\mathtt{IdentityToken}, token},}}
      \Statex \myss{\ \ \ \ \ \ \ \ \ \ \ \ \ \ \myangle{\mathtt{Key},key} \rangle ,\mathtt{s'.Parameters[RPDomain]}\rangle}
      \mystate{\myss{s'.q := stop}}
      \EndIf
      \ENDCASE 
  \ENDSWITCH
\mystate{\myss{\textbf{stop}\ \myangle{s',cookies,localStorage,sessionStorage,command}}}
    \end{algorithmic}
\end{breakablealgorithm}


\noindent\textbf{RP script}.

\begin{breakablealgorithm}
  \caption{$script\_rp$}
  \label{alg4}
  \begin{algorithmic}[1]
\Require {\myss{\langle tree, docnonce, scriptstate, scriptinputs, cookies, localStorage,}} 
\Statex \myss{sessionStorage, ids, secret\rangle}
\mystate{\myss{ s' := scriptstate}}
  \mystate{\myss{command := \myangle{}}}
  \mystate{\myss{IdPWindow := \mathtt{SUBWINDOW}(tree,docnonce).nonce}}
  \mystate{\myss{RPDomain := s'.RPDomain}}
  \SWITCH{\myss{s'.q}}
    \CASE{\myss{start}}
    \mystate{\myss{Url := \myangle{\mathtt{URL}, \mathtt{S}, RPDomain, /login, \myangle{}}}}
    \mystate{\myss{command := \myangle{\mathtt{IFRAME}, Url, \_SELF}}}
    \mystate{\myss{s'.q := expectReady}}
    \ENDCASE
    \CASE{\myss{expectReady}}
    \mystate{\myss{pattern := \myangle{\mathtt{POSTMESSAGE}, *, Content, *}}}
      \mystate{\myss{input := \mathtt{CHOOSEINPUT}(scriptinputs,pattern)}}
      \myif{input \not\equiv \mathtt{null} \land input.Content \equiv \mathtt{Ready}}
      \mystate{\myss{Content := \myangle{\mathtt{\mathtt{RPDomain}},RPDomain}}}
      \mystate{\myss{command := \myangle{\mathtt{POSTMESSAGE}, target, Content, \mathtt{null}}}}
      \mystate{\myss{s'.q := expectToken}}
      \EndIf
      \ENDCASE
      \CASE{\myss{expectToken}}
      \mystate{\myss{pattern := \myangle{\mathtt{POSTMESSAGE}, *, Content, *}}}
      \mystate{\myss{input := \mathtt{CHOOSEINPUT}(scriptinputs,pattern) }}
      \myif{input \not\equiv \mathtt{null}}
      \mystate{\myss{token := input.Content[Token]}}
      \mystate{\myss{key := input.Content[Token]}}
      \mystate{\myss{Url := \myangle{\mathtt{URL}, \mathtt{S}, RPDomain, /uploadToken, \myangle{}}}}
      \mystate{\myss{s'.refXHR :=  \mathtt{Random}()}}
      \mystate{\myss{command : = \langle \mathtt{XMLHTTPREQUEST}, Url, \mathtt{POST},}}
      \Statex \myss{\ \ \ \ \ \ \ \ \ \ \ \ \ \  \myangle{\myangle{\mathtt{Token}, token},\myangle{\mathtt{Key},key}}, s'.refXHR\rangle}
      \mystate{\myss{s'.q := stop}}
      \EndIf
      \ENDCASE
     \ENDSWITCH
\end{algorithmic}
\end{breakablealgorithm}



\noindent\textbf{Enclave application}.

\begin{breakablealgorithm}
  \caption{$R^e$}
  \label{alg2}
  \begin{algorithmic}[1]
  \Require {\myss{\myangle{a, f, m}, s}}
  \mystate{\myss{s':=s}}
  \mystate{\myss{\mathtt{InvokeFrom}:=f}}
  \myif{m.type \equiv \mathtt{ENCLAVEMESSAGE}}
  \mystate{\myss{token := m.Content[Token]}}
  \mystate{\myss{RPDomain := m.Content[RPDomain]}}
  \mystate{\myss{s'.Parameters[RPDomain]:=RPDomain}}
  \myif{token \not \equiv null}
  \mystate{\myss{n_1 = \mathtt{RANDOM}() }}
  \mystate{\myss{m' := \langle \mathtt{HTTPReq}, n_1, \mathtt{POST}, s'.IdP.Host, s'.IdP.UIDToken, }}
  \Statex \myss{\ \ \ \ \ \ \ \ \ \  \myangle{\myangle{\mathtt{UIDToken}, token}}\rangle}
  \mystop{s'.IdP.Address, \mathtt{\_SELF}, m'}
  \EndIf
  \EndIf
  \mystate{\myss{n, headers, body} \textbf{such that}} \myss{\myangle{\mathtt{HTTPResp},n, 200, headers,body} \equiv m}\\
  \ \ \textbf{if} \myss{possible}; \textbf{otherwise} stop \myss{\myangle{}, s'}
  \myif{n \equiv n_1}
  \mystate{\myss{uid :=body[UID]}}
	\myif{uid \not \equiv null}
	\mystate{\myss{key := \mathtt{GenerateKey()}}}	
	\mystate{\myss{PRPID := \mathtt{Encrypt}(s'.Parameters[RPDomain],key)}}
	\mystate{\myss{ PPID:=\mathtt{Encrypt}(\mathtt{Hash}(s'.Parameters[RPDomain],uid),key)}}
	\mystate{\myss{n_2 = \mathtt{RANDOM}() }}
	\mystate{\myss{m' := \langle \mathtt{HTTPReq}, n_2, \mathtt{POST}, s'.IdP.Host, s'.IdP.IdentityToken, }}
  \Statex \myss{\ \ \ \ \ \ \ \ \ \  \myangle{\myangle{\mathtt{PRPID}, PRPID},\myangle{\mathtt{PPID}, PPID}}\rangle}
  \mystop{s'.IdP.Address, \mathtt{\_SELF},  m'}
 \EndIf
 
 \myelse{n \equiv n_2}
 \mystate{\myss{token := body[Token]}}
 \mystate{\myss{m':=\myangle{\mathtt{ENCLAVEMESSAGE}, \myangle{\myangle{\mathtt{Token},token},\myangle{\mathtt{Key},key}}}}}
  \mystop{\mathtt{InvokeFrom}, \mathtt{\_SELF}, m'}
  \EndIf
  \mystop{}
  \end{algorithmic}
\end{breakablealgorithm}

