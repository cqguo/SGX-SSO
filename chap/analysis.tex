\section{Security Analysis}
\label{sec:analysis}
This section presents the analysis of security and privacy. 

\subsection{Security of UP-SSO}
We summarize the basic requirements of SSO systems based on existing theoretical analyses~\cite{ArmandoCCCT08,FettKS16, FettKS17} and practical attacks~\cite{SomorovskyMSKJ12,WangCW12,ArmandoCCCPS13,WangZLG16,MainkaMS16,MainkaMSW17,YangLCZ18}. These requirements enable an SSO system to provide secure authentication services for RPs.
\begin{itemize}
%\item
%\textbf{User identification.} When a user logs in to an RP multiple times,
 %the RP extracts the identical user identifier from identity proofs.
%    to provide personalized services for this user.

\item
\textbf{RP designation.} The designated RP is specified in identity proof,
    so that this identity proof is  only accepted by this RP.


\item
\textbf{Integrity and confidentiality.}
 Only the IdP can generate identity proofs,
and the identity proof with any modification would not be accepted.
Meanwhile, identity proof is only transmitted to the user and the designated RP.
\end{itemize}

%Although UP-SSO follows the main rules of OIDC, that RP redirects user to IdP, and IdP authenticates the user and returns an identity token back to RP. It also breaks OIDC's mechanisms satisfying the requirements of secure SSO. For example, in OIDC system, to guarantee the confidentiality of identity token, an identity token is transmitted to RP based on the HTTP redirection (i.e., 302 redirection) mechanism, and IdP guarantees the receiver's address belongs to honest RP (verifying \emph{redirect\_uri}). However, in UP-SSO, IdP does not learn the RP's identity, so that it needs to be proved that UP-SSO provides the secure mechanism to guarantee the confidentiality of identity token.

Following, we would describe the security mechanisms provided in UP-SSO system. 

%\vspace{1mm}\noindent\textbf{User identification.}
%UP-SSO does satisfy the requirement as it provides the PPID, from which the RP can derive a constant user account (i.e., the hash of PR domain and UID).%, similarly as it in OIDC system. 


\vspace{1mm}\noindent\textbf{RP designation.} 
%The idea of RP designation is that, IdP issues an identity token for user $Alice$ to visit RP $A$. Once $A$ receives this token, it cannot use the token to visit another RP $B$ as user $Alice$.
Different from identity token in OIDC system, the token issued by UP-SSO IdP contains the encrypted RP ID instead of plain one. 
We consider an adversary can get an identity token
, while the domain may belong to a malicious RP ($PRPID$ is the encrypted domain, and $PPID$ is the encrypted hash of the domain and user's $UID$, while the key is generated by enclave application).
% including $\\PRPID=enc(\mathtt{Malicious Domain})_{key}$ and $\\PPID=enc(hash(\mathtt{Malicious Domain}, UID))_{key}$.
There is no chance an adversary can find a key $s'$ satisfying that, the honest RP can derive its domain from $PRPID$, and derive the valid user account from $PPID$ with the same $s'$. 
To be noticed that, the valid user account must equals to the hash of honest RP's domain and honest user's $UID$.
%\newline $dec(PRPID)_{key'} \equiv \mathtt{Honest Domain}$ and  $\\ dec(PPID)_{key'}  \equiv  hash(\mathtt{Honest Domain}, UID)$.


\vspace{1mm}\noindent\textbf{Integrity of identity proof.}
%Although UP-SSO system offers the same identity token as it in OIDC system, 
An adversary may temper the identity proof by offering malicious key at step 22 in Figure~\ref{fig:UP-SSO}. 
We consider  only the IdP server and enclave application are honest. An adversary can get an identity proof,
while the $UID$ must belong to the malicious user, and the domain used for generating $PPID$ and $PRPID$ can be any registered RP domain.
There is no chance an adversary can get a key $s'$ and the domain satisfying that, 
the honest RP can derive its domain from $PRPID$, and derive the valid user account from $PPID$ with the same $s'$. 


\begin{comment}
including 
\newline $PRPID=enc(\mathtt{Any Domain})_{key}$ and 
\newline $PPID=enc(hash(\mathtt{Any Domain}, \mathtt{Malicious UID}))_{key}$. 
%The $Domain$ is controlled by adversary and can be assigned as any value. 
\newline The $Malicious UID$ must belong to the adversary, while the $\mathtt{Any Domain}$ can be any registered RP domain.
There is no chance an adversary can get $key'$  satisfying that, 
\newline $dec(PRPID)_{key'} \equiv \mathtt{Honest Domain}$ and  
\newline $dec(PPID)_{key'} \equiv  hash(\mathtt{Honest RPDomain}, \mathtt{Honest UID})$.
\end{comment}

Moreover, 
%unlike IdP in OIDC system retrieves the PPID from its database, IdP in UP-SSO system needs to receive the PPID from user client. Therefore, 
an adversary may try to lead IdP to issue a token with malicious PPID to break the integrity indirectly.  PPID is generated based on $Domain$ and $UID$, however, the domain is guaranteed by the $Cert$ and $UID$ is protected by the Key $K$. Thus the generation of $PPID$ cannot be undermined. Moreover, the $PPID$ is protected by $K$ during the transmission, so that it cannot be modified. Therefore, the integrity of identity proof is guaranteed. 
%Here we review Figure~\ref{fig:UP-SSO}. 
\begin{comment}
\item An adversary may attempt to temper the $PPID$ while it is transmitted between enclave application and IdP server. However, $PPID$ is encrypted (step 17) and the key is only known to enclave application and IdP server (key exchange at step 10).
\item The adversary may also try to undermine the generation of $PPID$. 
The $PPID$ is generated based on $RPDomain$ and $UID$ (step 17). According to the description in RP designation, the $RPDomain$ must belong to the target RP. The $UID$ is retrieved according to authentication result at step 6, and encrypted.
\item The adversary may also try to steal an honest user's encrypted $PPID$ and $UID$, and set them in step 16 or step 19. However, in an honest user's device, the IdP script and enclave application are honest, so that they do not send encrypted $PPID$ and $UID$ to any malicious party. Moreover, we consider honest user's device is secure, so that the adversary cannot steal $PPID$ and $UID$ in the manners, such as intercepting communications in user's device.
\end{comment}


\vspace{1mm}\noindent\textbf{Confidentiality of identity proof.}
Confidentiality of identity proof requires that the identity proof must be transmitted to RP correctly without being exposed to adversaries.
%As IdP in UP-SSO system does not learn the target RP's identity, it cannot sends the identity token to RP through an HTTP 302 redirection, which is widely used in SSO systems, such as OIDC. 

To guarantee the identity proof is securely transmitted, $postMessage$ restricted with an origin is adopted in UP-SSO system. That is, while the IdP script receives an identity proof from IdP server, it would invoke the $postMessage$ function provided by browser. And the target's origin is set as the value $RPDomain$ used in generating $PPID$ and $PRPID$. Browser guarantees that only the script belongs to the set origin can receive this identity proof.

%Even an adversary can work as a malicious script in user's browser and conduct malicious behaviour, such as man-in-the-middle attack, it cannot steal user's identity token. That is, while an adversary succeeds in tempering the RP's $Domain$ (at step 8 in Figure~\ref{fig:UP-SSO}), according to proof of RP designation, the identity token would not be accepted by an honest RP. Otherwise, while an adversary does not temper $Domain$, it cannot receive the identity token due to $postMessage$ mechanism.



\subsection{Privacy of UP-SSO}
According to Section~\ref{sec:design}, we can find that a curious IdP can only obtain the $PRPID$ related to the RP's identity. The $PRPID$ is the transformed RP domain, which is encrypted with a one-time key. Therefore, the IdP cannot know the real RP's identity or link multiple logins on the same RP. So the IdP-based identity tracing is not possible in UP-SSO system. 

Similarly, the RP can only obtain the $PPID$ from IdP. A malicious RP cannot derive the real user's UID from the $hash(Domain, UID)$. Thus, the collusive RPs are also unable to link the same user based on UID. Although the malicious RP can control the RP's $Domain$ to lead the IdP generate incorrect $PPID$, it still fails to accomplish the attack. Because due to the proof of confidentiality of identity proof, once the RP's $Domain$ is not consist with the RP's origin, the RP script would be unable to receive the identity proof. Therefore, the attack is not available. 

%However, the attack based on user's cookie is not considered in this paper. That is, the cookie of user may be exploited by malicious RPs to link a user at different RPs. For example, a user may visit multiple RPs at the same time. The malicious RPs may redirect the user to each other through the hidden iframe, carrying the $PPID$. This attack is also available in other user authentication systems besides of SSO system. Moreover, it can be easily detected through multiple methods, such as checking the iframe in the script, observing redirection flow through browser network tool, and detecting the redirection based on the browser extension.



























\begin{comment}
Our formal analysis of XXX is based on the Dolev-Yao style web model~\cite{SPRESSO}, which has been widely used in formal analysis of SSO protocol, e.g., OAuth 2.0~\cite{FettKS16} and OIDC ~\cite{FettKS17}.
To make the description cleaner, we foucus on our communications among XXX entities, and assume DNS and HTTPS are secure, which has already been anaylzed in~\cite{SPRESSO}.


\subsection{The Web Model}

The main entities in the model are $atomic\ processes$, which represent the essential nodes in the web systems, such as browsers, web servers and attackers. The atomic processes communicate with each other through the $events$ containing the receiver atomic process's address (e.g., IP), the sender atomic process process's address and the transmitted $messages$. Moreover, there are also dependent $scripting\ processes$ which runs on the client-side environment relying on the browsers such as JavaScript. The scripting provides the server defined function to the browser.  The web system mainly consists of the set of atomic processes and scripting processes. The operation of a system is described as that the system converts its states via step of runs. The state of web system is called $configuraton$ which consists of all the states of the atomic processes in the system and all the event can be accepted by the processes.

Here, we list the definitions of these notations as follows. 


\vspace{1mm}\noindent\textbf{Message } is defined as formal terms without variables (called ground terms). The messages in the model is considered containing constants (such as ASCII strings and nonce), sequence symbols (such as n-ary sequences $\langle \rangle$, $\langle . \rangle$, $\langle . ,. \rangle$ etc.) and further function symbols (such as encryption/decryption and digital signatures). For example, an HTTP request is a common message in the web model, containing a type $\mathtt{HTTPReq}$, a nonce $n$, a method  $\mathtt{GET}$ or $\mathtt{POST}$,  a domain , a path, URL parameters, request headers, and the body  in the sequence symbol formate. Here is an example for an HTTP GET request for the domain  $\mathtt{exa.com/path?para=1}$ with the headers and body empty.
\begin{equation*}
    m:=\langle\mathtt{HTTPReq},n,\mathtt{GET},exa.com,/path,\langle \langle para, 1\rangle \rangle ,\langle \rangle,\langle \rangle \rangle
\end{equation*}

\vspace{1mm}\noindent\textbf{Event } is the basic communication elements in the model.  An event is the term in the formate $\langle a, f, m \rangle$. In an $event$, the $f$ is the sender's address, the $a$ represents the address receiver, and $m$ is the message transmitted. 

\vspace{1mm}\noindent\textbf{Atomic Process}.  An $atomic\ Dolev-Yao (DY)\ process$ is can be displayed as the tuple $p=$ $(I^p, Z^p, R^p,s_0^p )$, which stands for the single node in the web model, such as the server and browser. $I^p$ includes the addresses owned by this process. $Z^p$ is the set of states that this process is probably in. $R^p$ is the set of relations between the pairs $\langle s, e \rangle$ and $\langle s', e' \rangle$ where $s, s' \in Z^p$.
That is, once a process is in the sate $s$ and receives the event $e$, it would jump into the state $s'$ and wait for the event $e'$.
%is the relation where inputting a state $s \in Z^p$ and an event $e$ and outputting a new state $s'$ and event $e'$, and $s_0 \in Z^p$ is the initial state of the process. 
%It's worth noting that for one process in a state only a finite set of events can be accepted by the process as the state and event are defined as the input of $R^p$.

\noindent\textbf{Browser Process.}
In XXX, we assume that the browsers are honest, therefore, we only need to analyze how the browsers interactive with the scripts. 
The browser model mainly includes the windows and documents.
\begin{itemize}
\item \noindent\textbf{Window}. The window \myss{w} represents the  the concrete browser window in the system, which is identified by a \myss{nonce}. A window contains a set of $\mathtt{documents}$ including the current document and cached documents. 
\item \noindent\textbf{Document}. The document is the HTML content in a window, which is identified by the document \myss{nonce}. It includes the scripting process downloaded from server.
\end{itemize}


\vspace{1mm}\noindent\textbf{Scripting Process}. The web model also contains the scripting process representing the client-side script loaded by browser such as JavaScript code. However, the $scripting\ process$ must rely on an $atom\ process$ such as browser and provide the relation $R$ witch is called by this $atomic\ process$. 

\vspace{1mm}\noindent\textbf{Equational Theory } is defined as usual in Dolev-Yao models,  but introduces the symbol $\equiv$ representing the congruence relation on terms. For instance,  $dec(enc(m,$ $ k),\ k)$ $\equiv$ $m$, where $k$ is the symmetric key.

\vspace{1mm}\noindent\textbf{Web System. }
The web system is consisted of a set of processes (including atomic processes and scripting processes). The web system can be described as the tuple ($\mathcal{W}$, $\mathcal{S}$, $\mathtt{script}$, $E^0$). $\mathcal{W}$ is consisted of the atomic processes, including honest processes and malicious processes. $\mathcal{S}$ is the set of scripting processes including honest scripts and malicious scripts. $\mathtt{script}$ is the set of concrete script code. And $E^0$ includes all the events that could be accepted by the processes in $\mathcal{W}$. 

\vspace{1mm}\noindent\textbf{Configuration}. 
In the web system, there is the set of states of all processes in $\mathcal{W}$ at one point in time, denoted as $S$. 
And all the $event$s can be accepted by the processes at this point consist the set $E$.
A $configuration$ of the system is defined as the tuple ($S, E, N$) where $N$ is the mentioned sequence of unused nonces. 

\vspace{1mm}\noindent\textbf{Run Step}. A run step is the process, that a web system changes its configuration ($S, E, N$) into ($S', E', N'$) after accepting an event $e \in E$. 

\subsection{Model Of XXX}
The XXX model is a web system which is defined as 
\begin{equation*}
    \mathcal{XWS} = (\mathcal{W}, \mathcal{S}, \mathtt{script}, E^0),
\end{equation*}

$\mathcal{W}$ is the finite set of atom processes in XXX system including a single IdP server process, multiple honest RP server processes, the browser processes, the enclave application processes and the attacker processes. We assume that all the honest RPs are implemented following the same rule so that the process are considered consistent besides of the addresses they listen to. The browsers controlled by user are considered honest. That is, the browser controlled by attackers can behave as  an independent atomic process. The enclave applications are always honest.

$\mathcal{S}$ is the finite set of scripting processes consists of $script\_rp$, $script\_idp$ and $script\_attacker$. The $script\_rp$ and $script\_idp$ are downloaded from honest RP and IdP processes and the $script\_attacker$ is downloaded from attacker process considered existing in all browser processes. 

\subsection{Security of XXX}
In this section, we prove the security of XXX. Here, we give the theorem to be proved. 
\begin{theorem}
Let $\mathcal{XWS}$ be the XXX web system, then $\mathcal{EWS}$ is secure.
\label{the:secure}
\end{theorem} 

The XXX is considered secure  {\sl iff} the adversary cannot log in to an honest RP under another honest user's account. Based on the model of XXX, described in Appendix~\ref{sec:appendix}, only when the RP accepts an valid identity proof and retrieves the PPID same as the honest user's from adversary, the attack is successful. Therefore, we can get the definition.

\begin{definition}
Let $\mathcal{XWS}$ be an XXX web system, $\mathcal{XWS}$ is secure \emph{iff} the adversary cannot obtain a valid identity token, whose encrypted PPID could be decrypted into the honest user's PPID.
\label{def:secure}
\end{definition}

To prove XXX system is secure, we only need to guarantee that it satisfies the requirements described in definition~\ref{def:secure}. 
The adversary may try to retrieve a valid identity token in following ways: (1) forging the valid identity token itself; (2) leading the honest RP treat adversary's identity token as the honest user's one; (3) stealing a valid token from an honest entity in the system. 
Therefore, definition~\ref{def:secure} can be further classified into the following lemmas. 

\begin{lemma}
The adversary cannot forge the valid identity token.
\label{lem:forge}
\end{lemma}

\begin{proof}
It can be easily proved that, while an RP receives the identity token, it verifies the signature with the public key of the IdP. Only the IdP knows the corresponding private key, so that the adversary cannot forge the valid identity token itself.
\end{proof}

\begin{lemma}
The RP would not retrieve an honest user's PPID from the adversary's identity token.
\label{lem:trickRP}
\end{lemma}

\begin{proof}
The adversary cannot obtain the identity proof issued for other honest user from XXX system, which is to be proved in lemma~\ref{lem:steal}. We consider in the SSO process, only the IdP server and enclave application are honest. An adversary can get a identity token including $\mathtt{PRPID}=encrypt(\mathtt{Domain})$ and $\mathtt{PRPID}=encrypt(hash(\mathtt{Domain}, \mathtt{uid}))$. The $\mathtt{Domain}$ is controlled by adversary and can be assigned as any value. The $\mathtt{uid}$ must belong to the adversary. There is no chance that an adversary can get a key, making the attack available. Because the key must satisfy that, $decrypt(\mathtt{PRPID}, key) \equiv honest\ RPDomain$ and  $decrypt(\mathtt{PPID}, key) \equiv hash(honest\ RPDomain, honest\ uid)$, which is not possible.   
\end{proof}


\begin{lemma}
The adversary cannot steal an identity token from any entities in the system.
\label{lem:steal}
\end{lemma}

\begin{proof}
We first give an outline of the proof. (1) For IdP, it only sends the identity token to its enclave application. (2) For enclave application, it only returns the token back to whom invoked it with the uid token. And it can be proved that only the honest user can own the uid token beside of enclave application and IdP server. (3) The IdP script only transmits the token to the script in the origin $\mathtt{RPDomain}$. As the identity token includes the $\mathtt{PRPID}$, the identity proof would be only sent to the origin that the token is issued for. (4) The RP script only sends the identity token to its server. (5) RP sever would not send identity token to any parties.
The detailed proof is described as follows. 

The identity token sent by IdP is shown in line 44, Algorithm~\ref{alg1}, as the response of path described in line 38, Algorithm~\ref{alg1}. We consider that IdP only accepts the enclave application's request to path $\mathtt{requireUID}$ and $\mathtt{requireToken}$. Therefore, the identity token would only be sent to the enclave application.

We can see that the identity token is sent by enclave application shown in line 25, Algorithm~\ref{alg5} to the entity defined in line 2, ~\ref{alg5}. The receiver of identity token is the one invoking it with a uid token (line 4, Algorithm~\ref{alg5}). And the uid is exchanged with this uid token (line 9, 14, Algorithm~\ref{alg5}). 
Then we prove that the only the honest user can own the uid token. Based on line 11-16 and line 22-28, Algorithm, we can see that IdP only sends the uid token to the authenticated user. It can be observed that the IdP script receives the uid token in line 34, Algorithm~\ref{alg3} and only sends it to enclave application in line 35, Algorithm~\ref{alg3}. Therefore, an adversary cannot obtain a user's uid token, so that it cannot retrieve the identity token from the enclave application.

The IdP script only sends the identity token by postMessage shown at line 43, Algorithm~\ref{alg3}. The target is the opener of this window and restricted by the origin $mathtt{RPDomain}$. $\mathtt{RPDomain}$ is defined at line 25, Algorithm~\ref{alg3} and never rewrote. Due to the scriptstate, in Algorithm~\ref{alg3} the $\mathtt{RPDomain}$s used at line 35 (used for identity token generation) and line 43 (restricting the receiver) must be consistent with the one at line 25. Therefore, IdP would not send the identity token to any adversaries. 

The model Algorithm~\ref{alg4} shows RP script receives the identity token at line 21 and send it out at line 25, while the receiver is defined at line 23. The receiver address is assigned during initiation at line 4 and never modified. It is downloaded with the script and considered honest. Therefore, the RP script would not send the identity token to adversary.

Based on the model shown as Algorithm~\ref{alg2}, we can find that RP server would not send identity token to other parties. Therefore, the adversary cannot steal the identity token from RP server.
\end{proof}

It is proved that XXX system satisfies the requirements raised in definition~\ref{def:secure}. Therefore, Theorem~\ref{the:secure} is proved.



\subsection{Privacy of UP-SSO}
Based on the process shown in Section~\ref{sec:design}, we can find that a curious IdP can only obtain the $PRPID$ related with the RP's identity. The $PRPID$ is the transformed RP domain, which is encrypted with an one-time key. Therefore, the IdP cannot know the real RP's identity or link multiple logins on the same RP. So the IdP-based identity tracing is not possible in UP-SSO system. 

Similarly, the RP can only obtains the $PPID$ from IdP. An malicious RP cannot derive the real user's UID from the $hash(RPDomain, UID)$. The collusive RPs are also unable to link the same user because of the same reason. Although the malicious RP can control the $RPDomain$ to lead the IdP generate incorrect $PPID$, it still fails to accomplish the attack. Because due to the proof of confidentiality, once the $RPDomain$ is not consist with the RP's origin, the RP script would be unable to receive the identity token. Therefore, the attack is not available. 

However, the attack based on user's cookie is not considered in this paper. That is, the cookie of user may be exploited by malicious RPs to link a user at different RPs. For example, a user may visit multiple RPs at the same time. The malicious RPs may redirect the user to each other through the hidden iframe, carrying the $PPID$. This attack is also available in other user authentication systems beside of SSO system. Moreover, it can be easily detected through multiple methods, such as checking the iframe in the script, observing redirection flow through browser network tool, and detecting the redirection based on the browser extension.

\end{comment}


