\section{Implementation and Evaluation}
\label{sec:implementation}
We have implemented  a prototype of UP-SSO.
%, containing an IdP server, an RP server and the user client. 
In this section, we describe the details of implementation and compare it with the open-source OIDC project to show its practical value.
\subsection{Implementation}
We adopt SHA-256 for digest generation, RSA-2048 for signature generation, and 128-bit AES/GCM algorithm for symmetrical encryption. 


IdP and RP servers are implemented based on Spring Boot, a popular web framework. 
We use the open-source auth0/java-jwt library to generate and verify the identity proof. 
The cryptographic computations of servers are implemented using API provided by JDK. 
Native messaging~\cite{NativeMessaging} (enables communication between chrome extension and native application) and chrome extension are  adopted to enable the JavaScript code to invoke the native application installed on user's device. 
%Native messaging offers the channel for chrome extension to communicate with native application. And chrome extension can communicate with the JavaScript code in the browser. 
%Many popular browsers have already supported native messaging, such as chrome and firefox. 
Enclave application is built based on Intel enclave SDK,
which offers the APIs for cryptographic computations.%, under hardware-level protection. 

\subsection{Evaluation}
\noindent\textbf{Environment.}
We deploy the IdP server on the device with Intel i7-8700 CPU and 32 GB RAM memory, running Windows 10 x64 system, and the RP server on the laptop with Intel i7-4700HQ CPU and 16 GB RAM memory, running Windows 10 x64 system. 
The browser is chrome deployed on the IdP device.  The devices are connected through the 1Gbps network. 



\vspace{1mm}\noindent\textbf{Result.}
We have run SSO process on our prototype system 100 times, and the average time is 171 ms (from user visiting an RP to RP returning the authentication result). As the remote attestation overhead is only counted once the enclave application is initiated, the overhead is not included (about 1139 ms). For comparison, we run MITREid Connect, a popular open-source OpenID Connect project. The time cost of  MITREid Connect is 113 ms. The overhead is modest.