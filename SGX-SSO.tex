% This is samplepaper.tex, a sample chapter demonstrating the
% LLNCS macro package for Springer Computer Science proceedings;
% Version 2.20 of 2017/10/04
%
\documentclass[conference]{IEEEtran}
%


% inlined bib file
\usepackage{filecontents}

\usepackage[compact]{titlesec}



%% Save the class definition of \subparagraph
\let\llncssubparagraph\subparagraph
%% Provide a definition to \subparagraph to keep titlesec happy
\let\subparagraph\paragraph
%% Load titlesec
%\usepackage[compact]{titlesec}
%% Revert \subparagraph to the llncs definition
\let\subparagraph\llncssubparagraph


%\usepackage{lipsum,mwe,cuted}
\usepackage{float}%%%%�ṩ�������[H]ѡ�����ȡ������
\usepackage{caption}%%�ṩ\captionof����


\pagestyle{plain}


\usepackage{graphicx}
\usepackage{verbatim}
\usepackage{caption}
%

\usepackage[noend]{algpseudocode}


\usepackage{amsmath}
\usepackage{amssymb}
%\usepackage{amsthm}

\usepackage{graphicx}
%\usepackage{geometry}
%\usepackage{subfigure}
\usepackage{url}
\usepackage{multirow}
\usepackage{listings}
\usepackage{cite}
\usepackage{array}
\usepackage{enumerate}
\usepackage{booktabs}
\usepackage{color}
\usepackage{soul}
\usepackage{multicol}
%\usepackage{algcompatible}

%\usepackage[compatible]{algpseudocode}

%\usepackage[algo2e]{algorithm2e}
\usepackage{algorithm}
%\usepackage{algorithmic}
% Used for displaying a sample figure. If possible, figure files should
% be included in EPS format.
%
% If you use the hyperref package, please uncomment the following line
% to display URLs in blue roman font according to Springer's eBook style:
% \renewcommand\UrlFont{\color{blue}\rmfamily}






\begin{document}
%
\title{UP-SSO: Enhancing the User Privacy of SSO by Integrating PPID and SGX
}%\thanks{Supported by organization x.}}
%
%\titlerunning{Abbreviated paper title}
% If the paper title is too long for the running head, you can set
% an abbreviated paper title here
%
\author{}
\begin{comment}
\author{First Author\inst{1}\orcidID{0000-1111-2222-3333} \and
Second Author\inst{2,3}\orcidID{1111-2222-3333-4444} \and
Third Author\inst{3}\orcidID{2222--3333-4444-5555}}
%
\authorrunning{F. Author et al.}
% First names are abbreviated in the running head.
% If there are more than two authors, 'et al.' is used.
%
\institute{Princeton University, Princeton NJ 08544, USA \and
Springer Heidelberg, Tiergartenstr. 17, 69121 Heidelberg, Germany
\email{lncs@springer.com}\\
\url{http://www.springer.com/gp/computer-science/lncs} \and
ABC Institute, Rupert-Karls-University Heidelberg, Heidelberg, Germany\\
\email{\{abc,lncs\}@uni-heidelberg.de}}
\end{comment}
%
\maketitle              % typeset the header of the contribution
%
\begin{abstract}
Single sign-on (SSO) services are widely deployed in the Internet as the identity management and authentication infrastructure.
In an SSO system, after authenticated by the identity provider (IdP),
 a user is allowed to log into relying parties (RPs) by presenting an identity proof.
Meanwhile, SSO introduces the potential leakage of user privacy:
    (\emph{a}) the curious IdP could track a user's all visits to any RP,
     and (\emph{b}) collusive RPs could link the user's identities across different RPs, to learn the user's network activity profile.
NIST suggests that pairwise pseudonymous identifiers (PPIDs) shall be adopted to prevent collusive RPs from linking a user,
    as PPID mechanism enables a user to log in multiple RPs with different identities.
However, the widely-adopted PPID mechanism cannot protect users from the IdP's tracking, as it still exposes RP identities to the IdP.

In this paper, we propose an SSO system, named \emph{UP-SSO},
 providing the enhanced PPID mechanism to protect a user's profile of RP visits from both the curious IdP and the collusive RPs by integrating PPID and SGX.
It separates the IdP's function into two parts, the server-side function and user-side part.
The generation of PPIDs is shifted from the IdP server to the user client, so that the IdP server no longer needs to learn RP identities to issue identity proofs.
%User client also transforms RP ID into a one-time encrypted ID, used by IdP for identity token generation, instead of RP's real ID.
The integrity at the user side is verified by the IdP through the remote attestation of SGX.
%The server-side service running on the IdP server still takes the responsibility of authenticating users, storing the private key, issuing the identity token, etc.
The detailed design of UP-SSO is described, and we analyze its security and privacy protection.
We implemented the prototype system of UP-SSO, and the evaluation shows the overhead is modest.
\end{abstract}
%
%
%


\section{Introduction}
\label{sec:intro}

%SSO的特点
%SSO的现状
Single sign-on (SSO) systems, such as OAuth, OpenID Connect and SAML, have been widely adopted nowadays as a convenient web authentication mechanism. SSO delegates user authentication on websites to a third party, so-called identity providers (IdPs), so that users can access different services on cooperating sites, so-called relying parties (RPs), via a single authentication process. Using SSO, a user no longer needs to maintain multiple credentials for different RPs, instead, she only maintains the credential for the IdP, which further provides identity proofs to RPs.
For RPs, SSO shifts the burden of user authentication to IdPs and therefore reduces their own security risks and costs. As a result, SSO has been widely integrated with modern web systems.
According to Alexa~\cite{Alexa}, 80 websites among the top-100 websites integrate SSO services. Meanwhile, many email and social networking providers such as Google, Facebook, Twitter, etc. have been actively serving as social identity providers to support social login.

However, the wide adoption of SSO also raises new privacy concerns. 
NIST~\cite{NIST2017draft} indicates that 
%Currently, more and more companies are interested in the users' online trace for business practices, such as accurate delivery advertising. For example, large internet service providers, such as Google and Facebook, are interested in collecting users' online behavioral information for various purposes (e.g., Screenwise Meter~\cite{googlenews} and Onavo~\cite{Onavo}). However, by simply serving the IdP role, these companies can easily collect a large amount of data to reconstruct users' online traces. 
curious IdP or multiple collusive RPs could break the users' privacy as follows.
\begin{itemize}
\item {\em IdP-based login tracing}. The IdP knows the identities of the RP and user in each single login instance, to generate the identity proof.
As a result, a curious IdP could discover all the RPs that the victim user attempts to visit and profile her online activities.
\item {\em RP-based identity linkage}. The RP learns a user's identity from the identify proof.
When the IdP generates identity proofs for a user, if the same user identifier is used in identity proofs generated for different RPs, malicious RPs could collude to not only link the user's login activities at different RPs for online tracking but also associate her attributes across multiple RPs.
\end{itemize}

The IdP-based login tracing and RP-based identity linkage would bring the real threat in reality. Imagine that, a user concerning her privacy would avoid to leave her full sensitive information at an application. She only leaves parts of her sensitive messages at each applications, for example, using real name on a social website and address on shopping website. And she would try not to leave any linkable message to avoid applications combining her informations, such as email.
% for example, if she leave the email on each applications, they can combine the parts of informations through the email. 
However, the privacy leaks in SSO systems make her effort in vain. As long as a user employs the SSO system, such as Google Account, to log in to these applications, the applications providers and Google can combine her information based on the SSO account.


To protect user privacy, the Pairwise Pseudonymous Identifier (PPID) is suggested by NIST~\cite{NIST2017draft} and adopted by some popular SSO protocols, such as OpenID Connect~\cite{OpenIDConnect,} and SAML~\cite{SAMLIdentifier}. 
In the PPID SSO system, IdP offers a user multiple individual IDs for different RPs. 
Therefore, an RP cannot  correlate the received  PPID  with the user's PPID at another RP. Thus, collusive RPs cannot link a user's logins from her PPIDs. 
Recently more and more IdPs have adopted PPID to protect user privacy. Moreover, Active Directory Federation Services and Oracle Access Management support the use of PPID, the privacy protection scheme suggested in OIDC protocol~\cite{MS, Oracle}. Identity service providers such as NORDIC APIS and CURITY suggest adopting PPID in SSO to protect user privacy~\cite{Nordic, Curity}.

There have already been many works on protecting user privacy in SSO systems. 
However, to the best of our knowledge, none of the existing techniques can be integrated into the existing PPID systems. 
Here we give a brief introduction of existing solutions for privacy-preserving SSO and explain the flaws of these schemes. 
\begin{itemize}
\item {\em Simply hiding RP ID from IdP. }For example, SPRESSO~\cite{SPRESSO} were proposed to defend against IdP-based login tracing by hiding RP ID from IdP. It uses the encrypted RP ID instead, and IdP issues the identity proof for this one-time ID. However, PPID is not available in this type of solution, as IdP cannot offer an RP the constant user ID with one-time irrelevant RP ID.
\item {\em Proving user identity based on zero-knowledge proof. }In this type of solution , such as EL PASSO~\cite{ZhangKSZR21} and UnlimitID~\cite{IsaakidisHD16}, the user needs to keep a secret $s$ and requires IdP to generate an identity proof for blinded $s$. Then user has to prove that she is the owner of $s$ to RP without exposing $s$ to RP. However, it is not convenient for user login on multiple devices. The user's identity is associated with the private $s$, therefore, if the user wants to log in to the RP on a new device, she must import the large $s$ into this device. For security consideration, the $s$ must be too long for user to remember. 
\item {\em Completely anonymous SSO system. }Anonymous SSO scheme is proposed to hide the user's identity to both the IdP and RPs with many methods. For example, in the anonymous SSO ~\cite{HanCSTW18}, it allows a user to visit the RP without exposing her identity to both IdP and RP based on zero-knowledge proof. However, it can only be applied to the anonymous services that do not identify the user, but not available in current personalized internet service.
\end{itemize}


%As discussed above, none of the existing SSO systems defend against both IdP-based login tracing and RP-based identity linkage, and provide the convenient SSO service on multiple devices at the same time.
%We remark that the problems cannot be solved, such as simply combining existing solutions together. The challenge is that, there is not a simple way for IdP to provide PPID without knowing the RP's identity.

In this paper, we propose Enhancing the User Privacy of SSO by Integrating PPID and SGX (named UP-SSO), the enhanced PPID SSO system with comprehensive protection against bosh IdP-based login tracing and RP-based identity linkage.
The key idea of UP-SSO is shifting part of IdP service, PPID generation, from server to user client. 
%Therefore, IdP server can issue the identity token containing PPID, only requiring an one-time transformed RP ID (e.g., encrypted RP ID). 
The user-side IdP service takes the responsibility for transforming RP ID (as an identity token must contain the RP ID to avoid misuse of token~\cite{YangLCZ18, WangZLG16, MainkaMS16, MainkaMSW17, YangLCZ18} ) and generating PPID. 
For IdP, it can only achieve a encrypted one-time RP ID, so that the IdP-based login tracing is not impossible.
To be noticed is that the user-side service must be deployed at the user controlled and IdP trusted environment, i.e., Intel SGX. With SGX, the secure hardware supported by intel CPU, the user-side service can be deployed at user's PC, and it can be verified by IdP through $remote attestation$, the function provided by SGX.
Moreover, other essential IdP services, such as user authentication and identity proof issuing,  must be completed at server side for security consideration. 

The overview of UP-SSO is that 


We summarize our contributions as follows.
%我们的贡献
%提出协议
%考虑能否根据模型进行分析
%实现原型系统
%The main contributions of UPPRESSO are as follows:
\begin{itemize}
\item We propose the comprehensive solution to hide the users' login traces from both the curious IdP and malicious collusive RPs for convenient SSO system.
\item We formally analyze the security of XXX and show that it guarantees the security, while the users' login traces are well protected.
\item We have implemented a prototype of XXX, and compare the performance of the XXX prototype with the state-of-the-art SSO systems (e.g., OIDC), and demonstrate its efficiency.
\end{itemize}

%文章结构
The rest of the paper is organized as follows. We first introduce the background in Section~\ref{sec:background}. Then, we describe the threat model, and our XXX design in Sections~\ref{sec:threatmodel} and \ref{sec:design}, followed by a formal analysis of security and privacy in Section~\ref{sec:analysis}. We provide the implementation specifics and experiment evaluation in Section~\ref{sec:implementation}, discuss the related works in Section~\ref{sec:relatedwork}, and conclude our work in Section~\ref{sec:conclusion}.

\section{Background}
\label{sec:background}
UP-SSO is compatible with OIDC, and achieves privacy protections based on Intel SGX.
%Here, we provide a brief introduction on OIDC and the SGX.
\subsection{OpenID Connect and PPID}
OIDC~\cite{OpenIDConnect} is an extension of OAuth 2.0 to support user authentication,
 and one of the most prominent SSO protocols.
%Same as other SSO protocols~\cite{SAMLIdentifier}, OIDC involves three entities, i.e., {\em users}, the {\em identity provider (IdP)}, and {\em relying parties (RPs)}.
%PPID is suggested in OpenID Connect protocol to protect the user from a possible correlation among RPs.

\begin{figure}[t]
  \centering
  \includegraphics[width=\linewidth]{fig/OIDC1.pdf}
  \vspace{-4mm}
  \caption{The implicit protocol flow of OIDC.}
  \label{fig:OpenID}
  \vspace{-4mm}
\end{figure}

\vspace{1mm}\noindent\textbf{Implicit Flow of User Login.}
The implicit flow is the typical user login flow of OIDC.
% ����ģʽ�Ͳ�д�ˣ��ڱ����У�д��û������
%OIDC supports three processes for the SSO authentication session, known as {\em implicit flow}, {\em authorization code flow} and {\em hybrid flow} (i.e., a mix-up of the previous two). Here, we choose the OIDC implicit flow as the example to illustrate the protocol.
%In the implicit flow, an {\em id token} is generated as the identity proof, which contains a user identifier, an RP identifier, IdP's identifier, the validity period, and other requested attributes.
%The IdP signs the id token using its private key to ensure integrity.
As shown in Figure~\ref{fig:OpenID},
at the beginning, a user attempts to log into an RP (Step 1).
The RP constructs a request for identity proofs,
 which is redirected by the user to the IdP.
This request contains $ID_{RP}$ and other optional attributes (Step 2).
If the user has not been authenticated, the IdP initiates an authentication process (Step 3).
Then, the IdP signs $ID_{RP}$ and the user's unique identity (or PPID) into an identity proof, and sends it back to the RP (Steps 4 and 5).
The RP finally verifies this identity proof, extracts the user identity, and returns the result.
\begin{comment}
\item[1] At the beginning, user attempts to log in to an RP.
\item[2] RP constructs a request for identity proof, which is redirected by the user to the corresponding IdP. The request contains $ID_{RP}$, RP's endpoint other attributes.
\item[3] If the user has not been authenticated yet, the IdP performs an authentication process.
\item[4] IdP signs the $ID_{RP}$ and PPID as the identity proof.%If the RP's endpoint in the request matches the one registered at the IdP, it generates an identity token. Otherwise, IdP generates a warning to notify the user about potential identity proof leakage.
\item[5] IdP sends the identity proof back to the RP.
\item[6] The RP verifies the id token, and extracts user identifier.
\item[7] Finally, RP returns the authentication result to the user.
\end{comment}

\vspace{1mm}\noindent\textbf{Pairwise Pseudonymous Identifier (PPID).}
%PPID is the user identifier provided by IdP identifying a user.
Different from the unique user identity at the IdP,
 a PPID is an RP-specific user ID.
That is, while a user visits different RPs, the IdP will provide different but constant user IDs for these RPs.
NIST~\cite{NIST2017draft} suggests that PPID should be either generated randomly and assigned to users by the IdP,
 or derived from other user's information if the derivation is done in an irreversible, unguessable manner.
%that PPID should be used to prevent the user's account at the IdP from being easily linked at multiple RPs through use of a common identifier.
%It also suggests that PPID should be either generated randomly and assigned to users by the IdP, or derived from other user's information if the derivation is done in an irreversible, unguessable manner (e.g., using a keyed hash function with a secret key).
%For example, we have learned that, MITREid Connect, a popular open-source OIDC project, generates PPID using a secure random number generator, and stores the key-value map of corresponding RP and PPID.


\subsection{Intel SGX}
Intel Software Guard Extensions (Intel SGX)~\cite{costan2016intel} is the hardware-based security  mechanism provided by Intel processors.  Enclaves~\cite{costan2016intel} and remote attestation~\cite{costan2016intel} mechanism are offered by Intel SGX.
%It offers memory encryption that isolates specific application code and data in memory.
%It allows user-level code to allocate private regions of memory, called enclaves~\cite{costan2016intel}, which guarantees the running codes are well protected from the adversary outside the enclave.

\vspace{1mm}\noindent\textbf{Enclave.}
The enclave’s code and data is stored in Processor Reserved Memory (PRM).
PRM cannot be directly accessed by other software, including system software and System Management Module code (Ring 2).
The Direct Memory Access (DMA) of PRM is also unavailable, as enclave is protected from other peripherals. That is, while the application is running inside the enclave, even the device owner cannot break its security, such as accessing or tempering the application's data.



\vspace{1mm}\noindent\textbf{Remote Attestation.}
Remote attestation enables the software inside an  enclave to attest to a remote entity that it is trusted.
%That is, during the attestation, the remote entity would receive an SGX attestation signature, containing the enclave’s measurement (a measurement of the code and data loaded in enclave).
The SGX remote attestation allows a player to verify the application's identity, intactness (never tampered), and that it is running securely within an enclave.
Moreover, with the remote attestation, the secure key exchange between the player and remote enclave application is also available.% even the application runs in a malicious environment.



\section{Threat Model and Assumptions}
\label{sec:threatmodel}
UP-SSO is compatible with OIDC, consisted of a number of RPs, user agents(including browser and enclave application) and an IdP. In this section, we describe the threat model and assumptions of these entities in UP-SSO.
\subsection{Threat Model}
\noindent\textbf{Adversaries' Goal: } 
\begin{itemize}
\item \noindent\textbf{Breaking the security. }The adversaries can impersonate an honest user to log in to the honest RP.
\item \noindent\textbf{Breaking the privacy. }The adversaries can track a user's login trace on each RP.
\end{itemize} 

\noindent\textbf{Adversaries' Capacities: }
\begin{itemize}
\item \noindent\textbf{Malicious user. }An adversary can act as the malicious user. The adversary can control all the software running outside the enclave, such as capturing and tempering the message transmission among enclave application and other entities, decrypting and tempering the https flow outside the enclave, tempering the script code running on the browser. For example, a malicious user may try to temper the PPID sent to IdP to achieve an identity token representing an honest user. 
\item \noindent\textbf{Malicious RP. }An adversary can act as the malicious RP. The adversary can lead the user to log in to the malicious RP. In this situation, the adversary can manipulate or all the message transmitted through RP, and collect all the flows received from user to link the user identity. 
\item \noindent\textbf{Curious but honest IdP. }An adversary can act as the curious but honest IdP. The curious IdP can store and analyze the received messages, and perform the timing attacks, attempting to achieve the IdP-based linkage. However, the honest IdP must process the requests of RP registration and identity proof correctly, provide honest script, and never collude with others (e.g., malicious RPs and users).
\item \noindent\textbf{External attackers. }Moreover, an adversary can collect all the network flow transmitted through a user. The adversary can also lead the user download the malicious script on her browser.
\end{itemize}

\subsection{Assumptions}
We assume the honest user's device is secure, for example, the user would not install any malicious application on her device. The application and data inside the enclave are never tempered or leaked, even in the malicious user's device.

The TLS is also adopted and correctly implemented at the system, so that the communications among entities ensure the confidentiality and integrity.
The cryptographic algorithms and building blocks used in UP-SSO are assumed to be secure and correctly implemented. 

Phishing attack is not considered in this paper.


\section{Design of XXX}
\begin{figure}
  \centering
  \includegraphics[width=\linewidth]{fig/sgx-sso.pdf}
  \caption{The protocol flow of XXX.}
  \label{fig:XXX}
\end{figure}
The XXX is compatible with OIDC, besides that the IdP service is separated into user agent part and server part. 
The server part IdP service takes the responsibility of authenticating the user, retrieves the UID for each user, and issues the signed identity proof consisted the privacy-preserving RP and user identifier generated at user agent part IdP service.
The user agent part IdP service would obtain the UID from server part service, transform UID into the PPID, and encrypt the RPID and PPID with an one-time symmetric key to avoid IdP server digging out the RP's identity. As the enclave application is protected by SGX, it must not conduct any malicious behaviour. 

\subsection{XXX process}
In this section, we provide the detail protocol of XXX.

The process of XXX is depicted in detail in Figure~\ref{fig:XXX}.
The SSO process is started with the user's visit to an RP at her browser, and the browser downloads the RP script (step 1), which is used to conduct the behaviour defined by RP at user side. Then the RP script opens the new window with the RP login endpoint (step 2, 3). Then the user is redirected to the IdP server (step 4). It must be noticed that, the user cannot visit IdP at step 2,3 directly because of the $Referer$ attribute in HTTP header. While the script in origin A opens a new window with origin B, the HTTP request to B will carry the key value $Referer: A$. Therefore, the RP's domain is exposed to IdP. With HTML5, a special attribute for links in HTML was introduced, that the $ref="noreferrer"$ can be used to make $Referer$ header be suppressed. However, when such a link is used to open a new window, the new window does not have a handle on the opening window (opener) anymore. The handle is necessary for XXX to transmit messages between RP and IdP.



While the user is redirected to the IdP server, the IdP will retrieve the IdP script (step 5), witch is used to deal with the interaction with IdP server, RP script and enclave application.
And the IdP authenticates the user (step 6). After the IdP script is downloaded, it sends the ready signal to its opener (i.e., the RP window) (step 7). Then it will receive the $RPdomain$ from RP script for further process (step 8). There are some types of parameters required in OIDC protocol to be carried in the SSO request, such as $response\_type$ and $scope$. In this paper, we would not focus on these attributes, and only describe the necessary parameters. 

Then the user starts the SSO request to IdP (step 9). As the result, the user receives a $UID token$ from IdP (step 10), with which an enclave application can retrieve the user's real identifier UID from IdP server.

Then the IdP script invokes the enclave application with RP's domain and the $UID token$ (step 11). The enclave application requires the UID from IdP server with the UID token (step 12, 13). After receiving the UID, enclave application generates the symmetric key s, encrypts the RP's domain with key s as the transformed RP ID, and encrypts the hash of RP's domain and UID as the PPID (step 14). 

After the ID transformation, enclave application requires the IdP server to generate identity proof with $PRPID$ and $PPID$ (step 15). The IdP server signs the $token$ consisted of $PRPID$ and $PPID$ as the identity proof (step 16), and return it to enclave application (step 17). Then enclave application set the $token$ and key $s$ as the result of step 11 (step 18). The IdP script then sends the $token$ and $s$ to the origin $RPdomain$ through $postMessage$ (step 19). It guarantees that only the script running in the RP window can receive the $token$, which avoids the man-in-the-middle attack.

Finally, the RP script uploads the $token$ and key $s$ to RP server (step 20). The RP server firstly verifies the signature with IdP's public key, then generates the $PRPID$ with its domain and key $s$, and compared it with the one carried by $token$. If the two $PRPID$s are equal, RP decrypts the user's ID from $PPID$, and find out the related user information in its database (step 21). Then it returns the login result to user (step 22). 
\section{Security Analysis}
\label{sec:analysis}
This section presents the analysis of security and privacy of UP-SSO. 

\subsection{Security of UP-SSO}
We summarize the basic requirements of SSO systems based on existing theoretical analyses~\cite{ArmandoCCCT08,FettKS16, FettKS17} and practical attacks~\cite{SomorovskyMSKJ12,WangCW12,ArmandoCCCPS13,ZhouE14,WangZLLYLG15,WangZLG16,YangLLZH16,MainkaMS16,MohsenS16,MainkaMSW17,YangLCZ18,YangLS17,ShiWL19}. These requirements enable an SSO system to provide secure authentication services for RPs,
    through identity proofs.
\begin{itemize}
\item
\textbf{User identification.} When a user logs into a certain RP multiple times by submitting identity proofs,
 the RP extracts the identical user identifier from these identity proofs,
    to provide personalized services for this user.

\item
\textbf{RP designation.} The designated receiver (or RP) is specified in an identity proof,
    so that this identity proof is accepted by the visited RP only.

\item
\textbf{Integrity and confidentiality.}
 Only the IdP is trusted to generate identity proofs,
 RPs do not accept an identity proof with any modification or a forged one.
Meanwhile, valid identity proof is transmitted only to the user and the designated RP.
\end{itemize}

Although UP-SSO follows the main rules of OIDC, that RP redirects user to IdP, and IdP authenticates the user and returns an identity token back to RP. It also breaks OIDC's mechanisms satisfying the requirements of secure SSO. For example, in OIDC system, to guarantee the confidentiality of identity token, an identity token is transmitted to RP based on the HTTP redirection (i.e., 302 redirection) mechanism, and IdP guarantees the receiver's address belongs to honest RP (verifying \emph{redirect\_uri}). However, in UP-SSO, IdP does not learn the RP's identity, so that it needs to be proved that UP-SSO provides the secure mechanism to guarantee the confidentiality of identity token.

In this section, we would describe the security mechanisms provided in UP-SSO system. 

\vspace{3mm}\noindent\textbf{User identification.}
UP-SSO does satisfy the requirement as it provides the PPID, from which the RP can derive a constant user account (i.e., the hash of PR domain and UID), similarly as it in OIDC system. 


\vspace{3mm}\noindent\textbf{RP designation.} 
The idea of RP designation is that, IdP issues an identity token for user $Alice$ to visit RP $A$. Once $A$ receives this token, it cannot use the token to visit another RP $B$ as user $Alice$.
 
Different from identity token in OIDC system, the token issued by UP-SSO IdP contains the encrypted RP ID instead of a plain RP ID. 
We consider an adversary can get a identity token including $PRPID=encrypt(Domain, key)$ and $PID=encrypt(hash(Domain, UID), key)$. The $Domain$ belongs to an adversary controlled server. The $UID$ belongs to an honest user. 
There is no chance that an adversary can find the key $key'$ satisfying that, $decrypt(encrypt\\(Domain, key), key') \equiv honest\ RPDomain$ and  $ decrypt(encrypt(hash(Domain, \\UID), key), key')  \equiv  hash(honest\ RPDomain, UID)$.



\vspace{3mm}\noindent\textbf{Integrity of identity token.}
Although UP-SSO system offers the same identity token as it in OIDC system, an adversary can even try to temper the identity token by providing a malicious key $s$ at step 22 in Figure~\ref{fig:UP-SSO}. 
We consider in the SSO process, only the IdP server and enclave application are honest. An adversary can get a identity token including $PRPID=encrypt(Domain, key)$ and $PPID=encrypt(hash(Domain, UID), key)$. The $Domain$ is controlled by adversary and can be assigned as any value. The $UID$ must belong to the adversary. There is no chance that an adversary can get a key, making the attack available. Because the key $key'$ must satisfy that, $decrypt(PRPID, key') \equiv honest\ RPDomain$ and  $decrypt(PPID, key') \equiv  hash(honest\ RPDomain, honest\ UID)$, which is not possible.

Moreover, unlike IdP in OIDC system retrieves the PPID from its database, IdP in UP-SSO system needs to receive the PPID from user client. Therefore, an adversary may try to lead IdP to accept a malicious PPID to break the integrity indirectly. 
Here we review Figure~\ref{fig:UP-SSO}. 
\begin{itemize}
\item An adversary may attempt to temper the $PPID$ while it is transmitted between enclave application and IdP server. However, $PPID$ is encrypted (step 17) and the key is only known to enclave application and IdP server (key exchange at step 10).
\item The adversary may also try to undermine the generation of $PPID$. 
The $PPID$ is generated based on $RPDomain$ and $UID$ (step 17). According to the description in RP designation, the $RPDomain$ must belong to the target RP. The $UID$ is retrieved according to authentication result at step 6, and encrypted.
\item The adversary may also try to steal an honest user's encrypted $PPID$ and $UID$, and set them in step 16 or step 19. However, in an honest user's device, the IdP script and enclave application are honest, so that they do not send encrypted $PPID$ and $UID$ to any malicious party. Moreover, we consider honest user's device is secure, so that the adversary cannot steal $PPID$ and $UID$ in the manners, such as intercepting communications in user's device.
\end{itemize}


\vspace{3mm}\noindent\textbf{Confidentiality of identity token.}
Confidentiality of identity token requires that the identity token must be transmitted to RP correctly without being exposed to adversaries.
As IdP in UP-SSO system does not learn the target RP's identity, it cannot sends the identity token to RP through an HTTP 302 redirection, which is widely used in SSO systems, such as OIDC. 

To guarantee the identity token is securely transmitted, $postMessage$ restricted with an origin is adopted in UP-SSO system. That is, while the IdP script receives an identity token from IdP server, it would invoke the $postMessage$ function provided by browser. And the target's origin is set as the value $RPDomain$ used in generating $PPID$ and $PRPID$. Browser guarantees that only the script belongs to the set origin can receive this identity token.

Even an adversary can work as a malicious script in user's browser and conduct malicious behaviour, such as man-in-the-middle attack, it cannot steal user's identity token. That is, while an adversary succeeds in tempering the RP's $Domain$ (at step 8 in Figure~\ref{fig:UP-SSO}), according to proof of RP designation, the identity token would not be accepted by an honest RP. Otherwise, while an adversary does not temper $Domain$, it cannot receive the identity token due to $postMessage$ mechanism.



\subsection{Privacy of UP-SSO}
Based on the process shown in Section~\ref{sec:design}, we can find that a curious IdP can only obtain the $PRPID$ related to the RP's identity. The $PRPID$ is the transformed RP domain, which is encrypted with a one-time key. Therefore, the IdP cannot know the real RP's identity or link multiple logins on the same RP. So the IdP-based identity tracing is not possible in UP-SSO system. 

Similarly, the RP can only obtain the $PPID$ from IdP. A malicious RP cannot derive the real user's UID from the $hash(Domain, UID)$. The collusive RPs are also unable to link the same user because of the same reason. Although the malicious RP can control the RP's $Domain$ to lead the IdP generate incorrect $PPID$, it still fails to accomplish the attack. Because due to the proof of confidentiality of identity token, once the RP's $Domain$ is not consist with the RP's origin, the RP script would be unable to receive the identity token. Therefore, the attack is not available. 

However, the attack based on user's cookie is not considered in this paper. That is, the cookie of user may be exploited by malicious RPs to link a user at different RPs. For example, a user may visit multiple RPs at the same time. The malicious RPs may redirect the user to each other through the hidden iframe, carrying the $PPID$. This attack is also available in other user authentication systems besides of SSO system. Moreover, it can be easily detected through multiple methods, such as checking the iframe in the script, observing redirection flow through browser network tool, and detecting the redirection based on the browser extension.



























\begin{comment}
Our formal analysis of XXX is based on the Dolev-Yao style web model~\cite{SPRESSO}, which has been widely used in formal analysis of SSO protocol, e.g., OAuth 2.0~\cite{FettKS16} and OIDC ~\cite{FettKS17}.
To make the description cleaner, we foucus on our communications among XXX entities, and assume DNS and HTTPS are secure, which has already been anaylzed in~\cite{SPRESSO}.


\subsection{The Web Model}

The main entities in the model are $atomic\ processes$, which represent the essential nodes in the web systems, such as browsers, web servers and attackers. The atomic processes communicate with each other through the $events$ containing the receiver atomic process's address (e.g., IP), the sender atomic process process's address and the transmitted $messages$. Moreover, there are also dependent $scripting\ processes$ which runs on the client-side environment relying on the browsers such as JavaScript. The scripting provides the server defined function to the browser.  The web system mainly consists of the set of atomic processes and scripting processes. The operation of a system is described as that the system converts its states via step of runs. The state of web system is called $configuraton$ which consists of all the states of the atomic processes in the system and all the event can be accepted by the processes.

Here, we list the definitions of these notations as follows. 


\vspace{1mm}\noindent\textbf{Message } is defined as formal terms without variables (called ground terms). The messages in the model is considered containing constants (such as ASCII strings and nonce), sequence symbols (such as n-ary sequences $\langle \rangle$, $\langle . \rangle$, $\langle . ,. \rangle$ etc.) and further function symbols (such as encryption/decryption and digital signatures). For example, an HTTP request is a common message in the web model, containing a type $\mathtt{HTTPReq}$, a nonce $n$, a method  $\mathtt{GET}$ or $\mathtt{POST}$,  a domain , a path, URL parameters, request headers, and the body  in the sequence symbol formate. Here is an example for an HTTP GET request for the domain  $\mathtt{exa.com/path?para=1}$ with the headers and body empty.
\begin{equation*}
    m:=\langle\mathtt{HTTPReq},n,\mathtt{GET},exa.com,/path,\langle \langle para, 1\rangle \rangle ,\langle \rangle,\langle \rangle \rangle
\end{equation*}

\vspace{1mm}\noindent\textbf{Event } is the basic communication elements in the model.  An event is the term in the formate $\langle a, f, m \rangle$. In an $event$, the $f$ is the sender's address, the $a$ represents the address receiver, and $m$ is the message transmitted. 

\vspace{1mm}\noindent\textbf{Atomic Process}.  An $atomic\ Dolev-Yao (DY)\ process$ is can be displayed as the tuple $p=$ $(I^p, Z^p, R^p,s_0^p )$, which stands for the single node in the web model, such as the server and browser. $I^p$ includes the addresses owned by this process. $Z^p$ is the set of states that this process is probably in. $R^p$ is the set of relations between the pairs $\langle s, e \rangle$ and $\langle s', e' \rangle$ where $s, s' \in Z^p$.
That is, once a process is in the sate $s$ and receives the event $e$, it would jump into the state $s'$ and wait for the event $e'$.
%is the relation where inputting a state $s \in Z^p$ and an event $e$ and outputting a new state $s'$ and event $e'$, and $s_0 \in Z^p$ is the initial state of the process. 
%It's worth noting that for one process in a state only a finite set of events can be accepted by the process as the state and event are defined as the input of $R^p$.

\noindent\textbf{Browser Process.}
In XXX, we assume that the browsers are honest, therefore, we only need to analyze how the browsers interactive with the scripts. 
The browser model mainly includes the windows and documents.
\begin{itemize}
\item \noindent\textbf{Window}. The window \myss{w} represents the  the concrete browser window in the system, which is identified by a \myss{nonce}. A window contains a set of $\mathtt{documents}$ including the current document and cached documents. 
\item \noindent\textbf{Document}. The document is the HTML content in a window, which is identified by the document \myss{nonce}. It includes the scripting process downloaded from server.
\end{itemize}


\vspace{1mm}\noindent\textbf{Scripting Process}. The web model also contains the scripting process representing the client-side script loaded by browser such as JavaScript code. However, the $scripting\ process$ must rely on an $atom\ process$ such as browser and provide the relation $R$ witch is called by this $atomic\ process$. 

\vspace{1mm}\noindent\textbf{Equational Theory } is defined as usual in Dolev-Yao models,  but introduces the symbol $\equiv$ representing the congruence relation on terms. For instance,  $dec(enc(m,$ $ k),\ k)$ $\equiv$ $m$, where $k$ is the symmetric key.

\vspace{1mm}\noindent\textbf{Web System. }
The web system is consisted of a set of processes (including atomic processes and scripting processes). The web system can be described as the tuple ($\mathcal{W}$, $\mathcal{S}$, $\mathtt{script}$, $E^0$). $\mathcal{W}$ is consisted of the atomic processes, including honest processes and malicious processes. $\mathcal{S}$ is the set of scripting processes including honest scripts and malicious scripts. $\mathtt{script}$ is the set of concrete script code. And $E^0$ includes all the events that could be accepted by the processes in $\mathcal{W}$. 

\vspace{1mm}\noindent\textbf{Configuration}. 
In the web system, there is the set of states of all processes in $\mathcal{W}$ at one point in time, denoted as $S$. 
And all the $event$s can be accepted by the processes at this point consist the set $E$.
A $configuration$ of the system is defined as the tuple ($S, E, N$) where $N$ is the mentioned sequence of unused nonces. 

\vspace{1mm}\noindent\textbf{Run Step}. A run step is the process, that a web system changes its configuration ($S, E, N$) into ($S', E', N'$) after accepting an event $e \in E$. 

\subsection{Model Of XXX}
The XXX model is a web system which is defined as 
\begin{equation*}
    \mathcal{XWS} = (\mathcal{W}, \mathcal{S}, \mathtt{script}, E^0),
\end{equation*}

$\mathcal{W}$ is the finite set of atom processes in XXX system including a single IdP server process, multiple honest RP server processes, the browser processes, the enclave application processes and the attacker processes. We assume that all the honest RPs are implemented following the same rule so that the process are considered consistent besides of the addresses they listen to. The browsers controlled by user are considered honest. That is, the browser controlled by attackers can behave as  an independent atomic process. The enclave applications are always honest.

$\mathcal{S}$ is the finite set of scripting processes consists of $script\_rp$, $script\_idp$ and $script\_attacker$. The $script\_rp$ and $script\_idp$ are downloaded from honest RP and IdP processes and the $script\_attacker$ is downloaded from attacker process considered existing in all browser processes. 

\subsection{Security of XXX}
In this section, we prove the security of XXX. Here, we give the theorem to be proved. 
\begin{theorem}
Let $\mathcal{XWS}$ be the XXX web system, then $\mathcal{EWS}$ is secure.
\label{the:secure}
\end{theorem} 

The XXX is considered secure  {\sl iff} the adversary cannot log in to an honest RP under another honest user's account. Based on the model of XXX, described in Appendix~\ref{sec:appendix}, only when the RP accepts an valid identity token and retrieves the PPID same as the honest user's from adversary, the attack is successful. Therefore, we can get the definition.

\begin{definition}
Let $\mathcal{XWS}$ be an XXX web system, $\mathcal{XWS}$ is secure \emph{iff} the adversary cannot obtain a valid identity token, whose encrypted PPID could be decrypted into the honest user's PPID.
\label{def:secure}
\end{definition}

To prove XXX system is secure, we only need to guarantee that it satisfies the requirements described in definition~\ref{def:secure}. 
The adversary may try to retrieve a valid identity token in following ways: (1) forging the valid identity token itself; (2) leading the honest RP treat adversary's identity token as the honest user's one; (3) stealing a valid token from an honest entity in the system. 
Therefore, definition~\ref{def:secure} can be further classified into the following lemmas. 

\begin{lemma}
The adversary cannot forge the valid identity token.
\label{lem:forge}
\end{lemma}

\begin{proof}
It can be easily proved that, while an RP receives the identity token, it verifies the signature with the public key of the IdP. Only the IdP knows the corresponding private key, so that the adversary cannot forge the valid identity token itself.
\end{proof}

\begin{lemma}
The RP would not retrieve an honest user's PPID from the adversary's identity token.
\label{lem:trickRP}
\end{lemma}

\begin{proof}
The adversary cannot obtain the identity proof issued for other honest user from XXX system, which is to be proved in lemma~\ref{lem:steal}. We consider in the SSO process, only the IdP server and enclave application are honest. An adversary can get a identity token including $\mathtt{PRPID}=encrypt(\mathtt{Domain})$ and $\mathtt{PRPID}=encrypt(hash(\mathtt{Domain}, \mathtt{uid}))$. The $\mathtt{Domain}$ is controlled by adversary and can be assigned as any value. The $\mathtt{uid}$ must belong to the adversary. There is no chance that an adversary can get a key, making the attack available. Because the key must satisfy that, $decrypt(\mathtt{PRPID}, key) \equiv honest\ RPDomain$ and  $decrypt(\mathtt{PPID}, key) \equiv hash(honest\ RPDomain, honest\ uid)$, which is not possible.   
\end{proof}


\begin{lemma}
The adversary cannot steal an identity token from any entities in the system.
\label{lem:steal}
\end{lemma}

\begin{proof}
We first give an outline of the proof. (1) For IdP, it only sends the identity token to its enclave application. (2) For enclave application, it only returns the token back to whom invoked it with the uid token. And it can be proved that only the honest user can own the uid token beside of enclave application and IdP server. (3) The IdP script only transmits the token to the script in the origin $\mathtt{RPDomain}$. As the identity token includes the $\mathtt{PRPID}$, the identity proof would be only sent to the origin that the token is issued for. (4) The RP script only sends the identity token to its server. (5) RP sever would not send identity token to any parties.
The detailed proof is described as follows. 

The identity token sent by IdP is shown in line 44, Algorithm~\ref{alg1}, as the response of path described in line 38, Algorithm~\ref{alg1}. We consider that IdP only accepts the enclave application's request to path $\mathtt{requireUID}$ and $\mathtt{requireToken}$. Therefore, the identity token would only be sent to the enclave application.

We can see that the identity token is sent by enclave application shown in line 25, Algorithm~\ref{alg5} to the entity defined in line 2, ~\ref{alg5}. The receiver of identity token is the one invoking it with a uid token (line 4, Algorithm~\ref{alg5}). And the uid is exchanged with this uid token (line 9, 14, Algorithm~\ref{alg5}). 
Then we prove that the only the honest user can own the uid token. Based on line 11-16 and line 22-28, Algorithm, we can see that IdP only sends the uid token to the authenticated user. It can be observed that the IdP script receives the uid token in line 34, Algorithm~\ref{alg3} and only sends it to enclave application in line 35, Algorithm~\ref{alg3}. Therefore, an adversary cannot obtain a user's uid token, so that it cannot retrieve the identity token from the enclave application.

The IdP script only sends the identity token by postMessage shown at line 43, Algorithm~\ref{alg3}. The target is the opener of this window and restricted by the origin $mathtt{RPDomain}$. $\mathtt{RPDomain}$ is defined at line 25, Algorithm~\ref{alg3} and never rewrote. Due to the scriptstate, in Algorithm~\ref{alg3} the $\mathtt{RPDomain}$s used at line 35 (used for identity token generation) and line 43 (restricting the receiver) must be consistent with the one at line 25. Therefore, IdP would not send the identity token to any adversaries. 

The model Algorithm~\ref{alg4} shows RP script receives the identity token at line 21 and send it out at line 25, while the receiver is defined at line 23. The receiver address is assigned during initiation at line 4 and never modified. It is downloaded with the script and considered honest. Therefore, the RP script would not send the identity token to adversary.

Based on the model shown as Algorithm~\ref{alg2}, we can find that RP server would not send identity token to other parties. Therefore, the adversary cannot steal the identity token from RP server.
\end{proof}

It is proved that XXX system satisfies the requirements raised in definition~\ref{def:secure}. Therefore, Theorem~\ref{the:secure} is proved.



\subsection{Privacy of UP-SSO}
Based on the process shown in Section~\ref{sec:design}, we can find that a curious IdP can only obtain the $PRPID$ related with the RP's identity. The $PRPID$ is the transformed RP domain, which is encrypted with an one-time key. Therefore, the IdP cannot know the real RP's identity or link multiple logins on the same RP. So the IdP-based identity tracing is not possible in UP-SSO system. 

Similarly, the RP can only obtains the $PPID$ from IdP. An malicious RP cannot derive the real user's UID from the $hash(RPDomain, UID)$. The collusive RPs are also unable to link the same user because of the same reason. Although the malicious RP can control the $RPDomain$ to lead the IdP generate incorrect $PPID$, it still fails to accomplish the attack. Because due to the proof of confidentiality, once the $RPDomain$ is not consist with the RP's origin, the RP script would be unable to receive the identity token. Therefore, the attack is not available. 

However, the attack based on user's cookie is not considered in this paper. That is, the cookie of user may be exploited by malicious RPs to link a user at different RPs. For example, a user may visit multiple RPs at the same time. The malicious RPs may redirect the user to each other through the hidden iframe, carrying the $PPID$. This attack is also available in other user authentication systems beside of SSO system. Moreover, it can be easily detected through multiple methods, such as checking the iframe in the script, observing redirection flow through browser network tool, and detecting the redirection based on the browser extension.

\end{comment}



%记得讲一下如何用js调用dll
\section{Implementation and Evaluation}
\label{sec:implementation}
\subsection{Implementation}
We have implemented  a prototype of XXX, containing an IdP server, an RP server and the enclave application. 
We adopt SHA-256 for digest generation, RSA-2048 for signature generation, and 128-bit AES/GCM algorithm for symmetrical encryption. 

\noindent\textbf{IdP server and RP server.}
IdP and RP servers are implemented based on Spring Boot, a popular web framework for the Java platform. We also use the open-source auth0/java-jwt library to generate and verify the identity token (in the form of Json Web Token). The encryption, decryption and other cryptographic computations are implemented using API provided by Java SDK. Moreover, the servers offer user interfaces to download JavaScript codes.

\noindent\textbf{Native Messaging.}
Beside of parties required in XXX system, we adopt chrome extension to enable the JavaScript code to invoke the enclave application. The chrome extension invokes a native application through the Native Messaging mechanism. It requires a user to update her Windows Registry by running a .reg profile. The profile defines the location of native application and which extension can invoke the application (identified by chrome-extension ID). 

\noindent\textbf{Enclave application.}
Intel provides the enclave SDK for developers to develop enclave applications for C++ platform. Enclave SDK includes the APIs for cryptographic computations, under hardware-level protection. An enclave application contains two parts, the inside and outside SGX parts. In our implementation, the outside part only takes responsibility of transmitting data between JavaScript code and inside-SGX-part enclave application.  

\subsection{Evaluation}
\noindent\textbf{Environment.}
In the evaluation, we deploy the RP and IdP servers on the device with Intel i7-8700 CPU and 32GB RAM memory, running Windows 10 x64 system. The browser is Chrome v90.0.4430.212, deployed on the same device.   



\noindent\textbf{Result.}
We have run SSO process on our prototype system 100 times, and the average time is 154 ms (excluding the overhead of remote attestation). For comparison, we run MITREid Connect, a popular open-source OpenID Connect implementation in Java. The time cost of  MITREid Connect is 113 ms. The overhead is modest.
%\input{chap/discussion.tex}
\section{Related Works}
\label{sec:relatedwork}
%\subsection{Security and Privacy of SSO system}
%Recently, SSO systems have been the essential infrastructure of internet service. Various SSO protocols, such as OAuth, OIDC and SAML are  widely adopted by Google, Facebook and other companies. There are also troubles about the security and privacy that are introduced with the SSO systems. 

\noindent\textbf{Security analysis of SSO systems.} Various attackers were found accessing the honest user's account at RP by multiple methods. 
Some attackers exploit the vulnerabilities of user's platforms to steal  identity proof (or cookie)~\cite{WangCW12,ArmandoCCCPS13}.
Some other attackers temper the identity proof to impersonate an honest user, such as XSW~\cite{SomorovskyMSKJ12}, RP's incomplete verification~\cite{WangCW12,WangZLG16,MainkaMSW17}, and IdP spoofing~\cite{MainkaMS16,MainkaMSW17}. 
Some IdP services did not bind the identity proof with a corresponding RP (or the honest RP did not verify the binding), so that malicious RP can leverage the received identity proof to access to another RP~\cite{YangLCZ18,WangZLG16,MainkaMS16,MainkaMSW17}. 
The formal analysis is also used to guarantee the security of SSO protocols~\cite{FettKS16,FettKS17}.
%Moreover, automatical tools, such as SSOScan~\cite{ZhouE14}, OAuthTester~\cite{YangLLZH16} and S3KVetter~\cite{YangLCZ18}, are also designed to detect vulnerabilities in SSO systems.

%Besides the analysis of the implemented SSO systems, the formal analysis is also used to guarantee the security of SSO protocols. Fett et al.~\cite{FettKS16, FettKS17} analyze the OAuth 2.0 and OIDC protocols based on the expressive Dolev-Yao style model~\cite{FettKS14}. The vulnerabilities found in these analyses enable the adversary steals the identity token form SSO systems. The analysis also shows that OAuth 2.0 and OIDC are secure once these two vulnerabilities are prevented. 

\vspace{1mm}\noindent\textbf{Privacy-preserving SSO systems.} NIST~\cite{NIST2017draft} suggests that the SSO should prevent user's trace from being tracked by both RP and IdP. 
%The pairwise user identifier (e.g., PPID) is used in some SSO protocols, such as SAML~\cite{SAML} and OIDC~\cite{OpenIDConnect}. 
Some protocols simply hide the RP's identity from IdP, such as SPRESSO~\cite{SPRESSO} (using encrypted RP identifier) and BrowserID~\cite{BrowserID} (RP identifier is added by user). However,  they are not defensive to RP-based identity linkage, and cannot integrate PPID. 
%Moreover, an analysis on Persona found IdP-based login tracing could still succeed~\cite{FettKS14, BrowserID}.
There is also another method that prevents both IdP and RP from knowing user's trace based on zero-knowledge algorithm, such as EL PASSO~\cite{ZhangKSZR21} and UnlimitID~\cite{IsaakidisHD16}. However, this type of solution requires user should remember a secret representing her identity, which is not convenient for login on multiple devices.

%Anonymous SSO schemes enable the users to access a service (i.e. RP) with permission from a verifier (i.e., IdP) without revealing their identities.
%The anonymous SSO systems are designed based on multiple methods. For example, various cryptographic primitives, such as group signature, zero-knowledge proof, etc., were used to design anonymous SSO schemes~\cite{WangWS13,HanCSTW18}. 
%The anonymous SSO schemes allow a user to obtain an unidentified token anonymously, and access RP's service with this token. However, the RP cannot classify a user's multiple logins, as the user is completely anonymous in this system. Therefore, they are not available in current personalized internet service.

%\subsection{Intel SGX}
%Intel SGX and remote attestation have been already used in many fields, particularly in authentication and building a trusted environment. 

\vspace{1mm}\noindent\textbf{Authentication systems built based on Intel SGX.} Intel SGX has been used in enhancing the security and privacy of authentication systems. Rafael et al.~\cite{CondeMW18} offer the credential protection approach based on SGX, which prevents an adversary from stealing a user's credential at server side. P2A~\cite{SongWLOWL20} is proposed to protect user's privacy. It enables a user to generate an identity proof locally while the registration, update, freeze/thaw, and deletion of identities are managed in a blockchain. Therefore, a user can visit a service without exposing her real identity. 

\begin{comment}
Remote attestation is wisely adopted on building the trusted environment. Jaehwan et al.~\cite{AhnLK20} propose a method
of building a safe smart home environment for IoT by pre-blocking devices based on remote attestation. 
Uzair et al.~\cite{JavaidAS20} establish the trust between IoT devices based on the blockchain and remote attestation, while the blockchain offers a secure framework for device registration. Moreover, there is work on how to conduct the remote attestation without an assigned key or secure memory (required in Intel SGX). Ilia et al.~\cite{LebedevHD18} propose the design of an attested execution processor based on RISC-V Rocket chip architecture. It does not 
require secure non-volatile memory, nor a private key explicitly assigned by the manufacturer.
\end{comment}
\section{Conclusion}
In this paper, we propose XXX, the XXX. To prevent the IdP from knowing a user's login trace in an SSO system, we split a IdP service into two parts. One part runs on the IdP server, which takes the responsibility of authenticating users, storing the private key, issuing the identity token and etc. The other part generates the PPID based on the user identifier retrieved by server-part service, and transforms PPID and RP's identifier into one-time encrypted ID to prevent both IdP-based user tracing and RP-based identity linkage. This part of service runs on the user's device, but protected by the Intel SGX. Even the malicious cannot control the user-part service. 
We formally analyse the XXX protocol based on Dolev-Yao style model, and make sure its security. 
The performance evaluation on the prototype of XXX demonstrates that the introduced is modest. 

\bibliographystyle{IEEEtran}
\bibliography{ref}

%\newpage
\appendix
\section{Appendix: Web Model}
%\begin{appendices}
\subsection{Message Format}
Here we provide the details of the format of the messages we use to construct the ExtraF model.

\vspace{1mm}\noindent\textbf{HTTP Messages}.
An HTTP request message is the term of the form
\vspace{-3mm}
\begin{equation*}
\langle \mathtt{HTTPReq}, nonce, method, host, path, parameters, headers, body\rangle
\end{equation*}

An HTTP response message is the term of the form
\vspace{-2mm}
\begin{equation*}
    \myangle{\mathtt{HTTPResp}, nonce, status, headers, body}
\end{equation*}
 The details are defined as follows:
 \begin{itemize}
 \item \myss{\mathtt{HTTPReq}} and \myss{\mathtt{HTTPResp}} denote the types of messages.
 \item \myss{nonce} is a random number that maps the response to the corresponding request.
 \item \myss{method} is the HTTP methods, such as \myss{\mathtt{GET}} and \myss{\mathtt{POST}}.
 \item \myss{host} is the constant string domain of visited server.
 \item \myss{path} is the constant string representing the concrete resource of the server.
 \item \myss{parameters} contains the parameters carried by the url as the form \\ \myss{\myangle{\myangle{name, value}, \myangle{name, value}, \dotsc}}, for example, the \myss{parameters} in the url \myss{http://www.example.com?type=confirm}  is \myss{\myangle{\myangle{type, confirm}}}.
 \item \myss{headers} is the header content of each HTTP messages as the form \\ \myss{\myangle{\myangle{name, value}, \myangle{name, value}, \dotsc}}, such as \\ \myss{\langle\myangle{Referer, http://www.example.com},} \myss{\myangle{Cookies, c}\rangle}.
 \item \myss{body} is the body content carried by HTTP \myss{\mathtt{POST}} request or HTTP response in the form \myss{\myangle{\myangle{name, value}, \myangle{name, value}, \dotsc}}.
  \item \myss{status} is the HTTP status code defined by HTTP standard.
 \end{itemize}

\vspace{1mm}\noindent\textbf{URL}.
URL is a term \myss{\myangle{\mathtt{URL}, protocol,host,path,parameters}}, where \myss{\mathtt{URL}} is the type, \myss{protocol} is chosen in \myss{\{\mathtt{S}}, \myss{\mathtt{P}\}} as \myss{\mathtt{S}} stands for HTTPS and \myss{\mathtt{P}} stands for HTTP. The \myss{host, path}, and \myss{parameters} are the same as in HTTP messages.

\vspace{1mm}\noindent\textbf{Origin}.
An Origin is a term \myss{\myangle{host, protocol}} that stands for the specific domain used by the HTTP CORS policy, where \myss{host} and \myss{protocol} are the same as in URL.

\vspace{1mm}\noindent\textbf{POSTMESSAGE}.
PostMessage is used in the browser for transmitting messages between scripts from different origins. We define the postMessage as the form \myss{\myangle{\mathtt{POSTMESSAGE}, target, Content, Origin}}, where \myss{\mathtt{POSTMESSAGE}} is the type, \myss{target} is the constant nonce which stands for the receiver, \myss{Content} is the message transmitted and \myss{Origin} restricts the receiver's origin.

\vspace{1mm}\noindent\textbf{XMLHTTPREQUEST}.
XMLHTTPRequest is the HTTP message transmitted  by scripts in the browser. That is, the XMLHTTPRequest is converted from the HTTP message by the browser. The XMLHTTPRequest in the form \myss{\myangle{\mathtt{XMLHTTPREQUEST}, URL, methods, Body, nonce}} can be converted into HTTP request message by the browser, and \myss{\myangle{\mathtt{XMLHTTPREQUEST}, Body, nonce}}  is converted from HTTP response message.

\vspace{1mm}\noindent\textbf{ENCLAVEMESSAGE}.
EnclaveMessage is the message transmitted between Script Process and Enclave Application. It is defined as the term \\ \myss{\myangle{\mathtt{ENCLAVEMESSAGE}, Content}}.


\noindent\textbf{Data Operation}.
The data used in ExtraF are defined in the following forms:
\begin{itemize}
\item \textbf{Standardized Data} is the data in the fixed format, for instance, the HTTP request is the standardized data in the form \myss{\langle\mathtt{HTTPReq}}, \myss{nonce}, \myss{method}, \myss{host}, \myss{path}, \myss{parameters}, \myss{headers}, \myss{body\rangle}.  We assume there is an HTTP request \myss{r :=}  \myss{\langle\mathtt{HTTPReq}},  \myss{n},  \myss{\mathtt{GET}},  \myss{example.com},  \myss{/path},  \myss{\myangle{}},  \myss{\myangle{}},  \myss{\myangle{}\rangle}, here we define the operation on the $r$. That is, the elements in $r$ can be accessed in the form \myss{r.name}, such that \myss{r.method \equiv \mathtt{GET}},  \myss{r.path \equiv /path} and \myss{r.body \equiv \myangle{}}.
\item \textbf{Dictionary Data} is the data in the form \myss{\myangle{\myangle{name, value}, \myangle{name, value}, \dotsc}}, for instance the \myss{body} in HTTP request is dictionary data. We assume there is a \myss{body := \myangle{\myangle{username, alice}, \myangle{password, 123}}}, here we define the operation on the $body$. That is, we can access the elements in \myss{body} in the form \myss{body[name]}, such that \myss{body[username] \equiv alice} and \myss{body[password] \equiv 123}.
%We can also add the new attributes to the dictionary, for example after we set \myss{body[age] := 18}, the \myss{body} are changed into\myss{ \myangle{\myangle{username, alice}, \myangle{password, 123}, \myangle{age, 18}}}.
\end{itemize}

\subsection{Browser Model}
In UPPRESSO, we assume that the browsers are honest, therefore, we only need to analyze how the browsers interactive with the scripts. 
We first introduce the windows and documents of the browser model.

\vspace{1mm}
\noindent\textbf{Window}. A window \myss{w} is a term of the form \myss{w = \myangle{nonce, documents, opener}}, representing the  the concrete browser window in the system. The \myss{nonce} is the window reference to identify each windows. The \myss{documents} is the set of documents (defined below) including the current document and cached documents (for example, the documents can be viewed via the ``forward" and ``back" buttons in the browser). The \myss{opener} represents the window in which this window is created, for instance, while a user clicks the href in document \myss{d} and it creates a new window \myss{w}, there is \myss{w.opener \equiv d.nonce}.

\vspace{1mm}
\noindent\textbf{Document}. A document \myss{d} is a term of the form
\begin{multline*}
\langle nonce, location, referrer, script, scriptstate, \\
 scriptinputs, subwindows, active \rangle 
\end{multline*}
where document is the HTML content in the window.  The \myss{nonce} locates the document. \myss{Location} is the URL where the document is loaded. \myss{Referrer} is same as the Referer header defined in HTTP standard. The \myss{script} is the scripting process downloaded from each servers. \myss{scriptstate} is define by the script, different in each scripts. The \myss{scriptinputs} is the message transmitted into the scripting process. The \myss{subwindows} is the set of \myss{nonce} of document's created windows. \myss{active} represents whether this document is active or not.

A scripting process is the dependent process relying on the browser, which can be considered as a relation \myss{R} mapping a message input and a message output. And finally the browser will conduct the command in the output message. Here we give the description of the form of input and output.
\begin{itemize}
\setlength\itemsep{-2pt}
\item \textbf{Scripting Message Input. } The input is the term in the form
\begin{multline*}
\langle tree, docnonce, scriptstate, stateinputs,cookies,\\
localStorage, sessionStorage, ids, secret \rangle
\end{multline*}
\item \textbf{Scripting Message Output. }The output is the term in the form
\begin{equation*}
\langle scriptstate, cookies, localStorage,sessionStorage, command \rangle
\end{equation*}
\end{itemize}
The \myss{tree} is the relations of the opened windows and documents, which are visible to this script. \myss{Docnonce} is the document nonce. The  \myss{Scriptstate} is a term of the form defined by each script. \myss{Scriptinputs} is the message transmitted to script. However, the \myss{scriptinputs} is defined as standardized forms, for example, postMessage is one of the forms of \myss{scriptinputs}. \myss{Cookies} is the set of cookies that belong to the document's origin. \myss{LocalStorage} is the storage space for browser and \myss{sessionStorage} is the space for each HTTP sessions.  \myss{Ids} is the set of user IDs while \myss{secret} is the password to corresponding user ID. The \myss{command} is the operation which is to be conducted by the browser. Here we only introduce the form of commands used in XXX system. We have defined the postMessage and XMLHTTPRequest (for HTTP request) message which are the \myss{commands}. Moreover, a term in the form \myss{\myangle{\mathtt{IFRAME}, URL, WindowNonce}} asks the browser to create this document's subwindow and it visits the server with the URL.

\subsection{XXX Model}

\noindent\textbf{IdP server}.

\begin{breakablealgorithm}
  \caption{$R^i$}
  \label{alg1}
  \begin{algorithmic}[1]
  \Require{\myss{\myangle{a, f, m}, s}}
  \mystate{\myss{s':=s}}
  \mystate{\myss{n, method, path, parameters, headers, body} \textbf{such that}}\\
  \ \ \myss{\myangle{\mathtt{HTTPReq},n,method,path,parameters,headers,body} \equiv m}\\
  \ \ \textbf{if} \myss{possible}; \textbf{otherwise} stop \myss{\myangle{}, s'}
  \myif{path \equiv /script}
  \mystate{\myss{m':=\myangle{\mathtt{HTTPResp},n,200, \myangle{}, \mathtt{IdPScript}}}}
  \mystop{f, a, m'}
  
  \myelse{path \equiv /login}
  \mystate{\myss{cookie := headers[Cookie]}}
  \mystate{\myss{session := s'.sessions[cookie]}}
  \mystate{\myss{username:=body[username]}}
  \mystate{\myss{password:=body[password]}}
  \myif{password \not\equiv \mathtt{SecretOfID}(username)}
  \mystate{\myss{m' :=\myangle{\mathtt{HTTPResp},n,200,\myangle{},\mathtt{LoginFailure}}}}
  \mystop{f,a,m'}
  \EndIf
  \mystate{\myss{session[uid] := \mathtt{UIDOfUser}(username)}}
  \mystate{\myss{m' :=\myangle{\mathtt{HTTPResp},n,200,\myangle{},\mathtt{LoginSucess}}}}
  \mystop{f,a,m'}
  
  \myelse{path \equiv /requireUIDToken}
  \mystate{\myss{cookie := headers[Cookie]}}
  \mystate{\myss{session := s'.sessions[cookie]}}
  \mystate{\myss{uid := session[uid]}}
  \myif{uid \equiv \mathtt{null}}
  \mystate{\myss{m' := \myangle{\mathtt{HTTPResp},n,200,\myangle{},\mathtt{UnLogged}}}}
  \mystop{f,a,m'}
  \EndIf
  \mystate{\myss{token := \mathtt{GenerateToken}()}}
  \mystate{\myss{s'.Tokens := s'.Tokens + ^{\myangle{}}\myangle{uid, token}}}
  \mystate{\myss{m' := \myangle{\mathtt{HTTPResp},n,200,\myangle{},\myangle{\mathtt{Token}, token}}}}
  \mystop{f,a,m'}
  
  \myelse{path \equiv /requireUID}
  \mystate{\myss{UIDToken := body[UIDToken]}}
  \mystate{\myss{uid := \mathtt{FindUIDByToken(UIDToken)}}}
  \myif{uid \equiv null}
  \mystate{\myss{m' := \myangle{\mathtt{HTTPResp}, n, 200, \myangle{}, \mathtt{TokenError}}}}
  \mystop{f,a,m'}
  \EndIf
  \mystate{\myss{m' := \myangle{\mathtt{HTTPResp}, n, 200, \myangle{}, \myangle{\mathtt{UID}, uid}}}}
  \mystop{f,a,m'}
  
  \myelse{path \equiv /requireToken}
  \mystate{\myss{PRPID := body[PRPID]}}
  \mystate{\myss{PPID := body[PPID]}}
  \mystate{\myss{Content := \myangle{PRPID,PPID}}}
  \mystate{\myss{token := Content+Sign(Content,s'.SignKey)}}
  \mystate{\myss{m' := \myangle{\mathtt{HTTPResp}, n, 200, \myangle{}, \myangle{\mathtt{Token},token}}}}
  \mystop{f,a,m'}
  \EndIf
  \mystop{}
  \end{algorithmic}
\end{breakablealgorithm}

\noindent\textbf{RP server}.

\begin{breakablealgorithm}
  \caption{$R^r$}
  \label{alg2}
  \begin{algorithmic}[1]
  \Require {\myss{\myangle{a, f, m}, s}}
  \mystate{\myss{s':=s}}
  \mystate{\myss{n, method, path, parameters, headers, body} \textbf{such that}}\\
  \ \ \myss{\myangle{\mathtt{HTTPReq},n,method,path,parameters,headers,body} \equiv m}\\
  \ \ \textbf{if} \myss{possible}; \textbf{otherwise} stop \myss{\myangle{}, s'}
  \myif{path \equiv /script}
\mystate{\myss{m':=\myangle{\mathtt{HTTPResp},n,200, \myangle{}, \mathtt{RPScript}}}}
  \mystop{f, a, m'}
  
  \myelse{path \equiv /uploadToken}
  \mystate{\myss{cookie := headers[Cookie]}}
  \mystate{\myss{session := s'.sessions[cookie]}}
  \mystate{\myss{token := body[Token]}}
  \mystate{\myss{key := body[Key]}}
  \myif{\mathtt{CheckSig}(Token.Content, Token.Sig, s'.IdP.PubKey)}
  \mystate{\myss{PRPID := \mathtt{Encrypt}(s'.RPDomain,key)}}  
  \myif{PRPID \equiv token.Content.PRPID}
  \mystate{\myss{PPID := token.Content.PPID}}
  \myif{PPID \not \in \mathtt{ListOfUser}()}
  \mystate{\myss{\mathtt{RegisterUser}(PPID)}}
  \EndIf
  \mystate{\myss{session[user] := PPID}}
  \mystate{\myss{m' := \myangle{\mathtt{HTTPResp}, n, 200, \myangle{}, \mathtt{LoginSuccess}}}}
  \EndIf
  \EndIf
  \mystate{\myss{m' := \myangle{\mathtt{HTTPResp}, n, 200, \myangle{}, \mathtt{Fail}}}}
  \mystop{f, a, m'}
  \EndIf
  \mystop{}
  \end{algorithmic}
\end{breakablealgorithm}


\noindent\textbf{IdP script}.

\begin{breakablealgorithm}
  \caption{$script\_idp$}
  \label{alg3}
  \begin{algorithmic}[1]
  \Require{\myss{\langle tree, docnonce, scriptstate, scriptinputs, cookies, localStorage,}}
  \Statex {\myss{ sessionStorage, ids, secret \rangle}}  
  \mystate{\myss{ s' := scriptstate}}
  \mystate{\myss{command := \myangle{}}}
  \mystate{\myss{target := \mathtt{PARENTWINDOW}(tree,docnonce)}}
  \mystate{\myss{IdPDomain := s'.IdPDomain}}
  \SWITCH{\myss{s'.q}}
    \CASE{\myss{startLogin}}
    \mystate{\myss{username \in ids}}
    \mystate{\myss{Url := \myangle{\mathtt{URL}, \mathtt{S}, IdPDomain, /login, \myangle{}}}}
    \mystate{\myss{s'.refXHR :=  \mathtt{Random}()}}
    \mystate{\myss{command : = \langle \mathtt{XMLHTTPREQUEST}, Url, \mathtt{POST},}}  
    \Statex \myss{\ \ \ \ \ \ \ \ \ \ \myangle{\myangle{username, username}, \myangle{password, secret}}, s'.refXHR \rangle}
    \mystate{\myss{s'.q := expectLoginResult}}
    \ENDCASE
    
    \CASE{expectLoginResult}
      \mystate{\myss{pattern := \myangle{\mathtt{XMLHTTPREQUEST},Body,s'.refXHR}}}
      \mystate{\myss{input := \mathtt{CHOOSEINPUT}(scriptinputs,pattern) }}
      \myif{input \not\equiv \mathtt{null}}
      \myif{input.Body \not\equiv \mathtt{LoginSuccess}}
      \mystate{\myss{\textbf{stop}\ \myangle{}}}
      \EndIf
      \mystate{\myss{command := \myangle{\mathtt{POSTMESSAGE}, target, \mathtt{Ready}, \mathtt{null}}}}
      \mystate{\myss{s'.q := expectRPDomain}}
      \EndIf
      \ENDCASE
    
    \CASE{\myss{expectRPDomain}}
      \mystate{\myss{pattern := \myangle{\mathtt{POSTMESSAGE}, *, Content, *}}}
      \mystate{\myss{input := \mathtt{CHOOSEINPUT}(scriptinputs,pattern)}}
      \myif{input \not\equiv \mathtt{null}}
      \mystate{\myss{RPDomain := input.Content[RPDomain]}}
      \mystate{\myss{s'.Parameters[RPDomain] := RPDomain}}
      \mystate{\myss{Url := \myangle{\mathtt{URL}, \mathtt{S}, IdPDomain, /requireUIDToken,\myangle{} }}}
      \mystate{\myss{s'.refXHR :=  \mathtt{Random}()}}
      \mystate{\myss{command : = \langle\mathtt{XMLHTTPREQUEST}, Url, \mathtt{GET}, s'.refXHR\rangle}}
      \mystate{\myss{s'.q := expectUIDToken}}
       \EndIf
      \ENDCASE
      
    \CASE{expectUIDToken}
      \mystate{\myss{pattern := \myangle{\mathtt{XMLHTTPREQUEST},Body,s'.refXHR}}}
      \mystate{\myss{input := \mathtt{CHOOSEINPUT}(scriptinputs,pattern) }}
      \myif{input \not\equiv \mathtt{null}}
      \mystate{\myss{token := input.Body[UIDToken]}}
      \mystate{\myss{command := \myangle{\mathtt{ENCLAVEMESSAGE}, \myangle{\mathtt{UIDToken}, token}}}}
      \mystate{\myss{s'.q := expectIdentityToken}}
      \EndIf
      \ENDCASE
    
    \CASE{expectIdentityToken}
      \mystate{\myss{pattern := \myangle{\mathtt{ENCLAVEMESSAGE},Content}}}
      \mystate{\myss{input := \mathtt{CHOOSEINPUT}(scriptinputs,pattern) }}
      \myif{input \not\equiv \mathtt{null}}
      \mystate{\myss{token:=input.Content[Token]}}
      \mystate{\myss{key:=input.Content[Key]}}
      \mystate{\myss{command := \langle\mathtt{POSTMESSAGE}, target, \langle\myangle{\mathtt{IdentityToken}, token},}}
      \Statex \myss{\ \ \ \ \ \ \ \ \ \ \ \ \ \ \myangle{\mathtt{Key},key} \rangle ,\mathtt{s'.Parameters[RPDomain]}\rangle}
      \mystate{\myss{s'.q := stop}}
      \EndIf
      \ENDCASE 
  \ENDSWITCH
\mystate{\myss{\textbf{stop}\ \myangle{s',cookies,localStorage,sessionStorage,command}}}
    \end{algorithmic}
\end{breakablealgorithm}


\noindent\textbf{RP script}.

\begin{breakablealgorithm}
  \caption{$script\_rp$}
  \label{alg4}
  \begin{algorithmic}[1]
\Require {\myss{\langle tree, docnonce, scriptstate, scriptinputs, cookies, localStorage,}} 
\Statex \myss{sessionStorage, ids, secret\rangle}
\mystate{\myss{ s' := scriptstate}}
  \mystate{\myss{command := \myangle{}}}
  \mystate{\myss{IdPWindow := \mathtt{SUBWINDOW}(tree,docnonce).nonce}}
  \mystate{\myss{RPDomain := s'.RPDomain}}
  \SWITCH{\myss{s'.q}}
    \CASE{\myss{start}}
    \mystate{\myss{Url := \myangle{\mathtt{URL}, \mathtt{S}, RPDomain, /login, \myangle{}}}}
    \mystate{\myss{command := \myangle{\mathtt{IFRAME}, Url, \_SELF}}}
    \mystate{\myss{s'.q := expectReady}}
    \ENDCASE
    \CASE{\myss{expectReady}}
    \mystate{\myss{pattern := \myangle{\mathtt{POSTMESSAGE}, *, Content, *}}}
      \mystate{\myss{input := \mathtt{CHOOSEINPUT}(scriptinputs,pattern)}}
      \myif{input \not\equiv \mathtt{null} \land input.Content \equiv \mathtt{Ready}}
      \mystate{\myss{Content := \myangle{\mathtt{\mathtt{RPDomain}},RPDomain}}}
      \mystate{\myss{command := \myangle{\mathtt{POSTMESSAGE}, target, Content, \mathtt{null}}}}
      \mystate{\myss{s'.q := expectToken}}
      \EndIf
      \ENDCASE
      \CASE{\myss{expectToken}}
      \mystate{\myss{pattern := \myangle{\mathtt{POSTMESSAGE}, *, Content, *}}}
      \mystate{\myss{input := \mathtt{CHOOSEINPUT}(scriptinputs,pattern) }}
      \myif{input \not\equiv \mathtt{null}}
      \mystate{\myss{token := input.Content[Token]}}
      \mystate{\myss{key := input.Content[Token]}}
      \mystate{\myss{Url := \myangle{\mathtt{URL}, \mathtt{S}, RPDomain, /uploadToken, \myangle{}}}}
      \mystate{\myss{s'.refXHR :=  \mathtt{Random}()}}
      \mystate{\myss{command : = \langle \mathtt{XMLHTTPREQUEST}, Url, \mathtt{POST},}}
      \Statex \myss{\ \ \ \ \ \ \ \ \ \ \ \ \ \  \myangle{\myangle{\mathtt{Token}, token},\myangle{\mathtt{Key},key}}, s'.refXHR\rangle}
      \mystate{\myss{s'.q := stop}}
      \EndIf
      \ENDCASE
     \ENDSWITCH
\end{algorithmic}
\end{breakablealgorithm}



\noindent\textbf{Enclave application}.

\begin{breakablealgorithm}
  \caption{$R^e$}
  \label{alg2}
  \begin{algorithmic}[1]
  \Require {\myss{\myangle{a, f, m}, s}}
  \mystate{\myss{s':=s}}
  \mystate{\myss{\mathtt{InvokeFrom}:=f}}
  \myif{m.type \equiv \mathtt{ENCLAVEMESSAGE}}
  \mystate{\myss{token := m.Content[Token]}}
  \mystate{\myss{RPDomain := m.Content[RPDomain]}}
  \mystate{\myss{s'.Parameters[RPDomain]:=RPDomain}}
  \myif{token \not \equiv null}
  \mystate{\myss{n_1 = \mathtt{RANDOM}() }}
  \mystate{\myss{m' := \langle \mathtt{HTTPReq}, n_1, \mathtt{POST}, s'.IdP.Host, s'.IdP.UIDToken, }}
  \Statex \myss{\ \ \ \ \ \ \ \ \ \  \myangle{\myangle{\mathtt{UIDToken}, token}}\rangle}
  \mystop{s'.IdP.Address, \mathtt{\_SELF}, m'}
  \EndIf
  \EndIf
  \mystate{\myss{n, headers, body} \textbf{such that}} \myss{\myangle{\mathtt{HTTPResp},n, 200, headers,body} \equiv m}\\
  \ \ \textbf{if} \myss{possible}; \textbf{otherwise} stop \myss{\myangle{}, s'}
  \myif{n \equiv n_1}
  \mystate{\myss{uid :=body[UID]}}
	\myif{uid \not \equiv null}
	\mystate{\myss{key := \mathtt{GenerateKey()}}}	
	\mystate{\myss{PRPID := \mathtt{Encrypt}(s'.Parameters[RPDomain],key)}}
	\mystate{\myss{ PPID:=\mathtt{Encrypt}(\mathtt{Hash}(s'.Parameters[RPDomain],uid),key)}}
	\mystate{\myss{n_2 = \mathtt{RANDOM}() }}
	\mystate{\myss{m' := \langle \mathtt{HTTPReq}, n_2, \mathtt{POST}, s'.IdP.Host, s'.IdP.IdentityToken, }}
  \Statex \myss{\ \ \ \ \ \ \ \ \ \  \myangle{\myangle{\mathtt{PRPID}, PRPID},\myangle{\mathtt{PPID}, PPID}}\rangle}
  \mystop{s'.IdP.Address, \mathtt{\_SELF},  m'}
 \EndIf
 
 \myelse{n \equiv n_2}
 \mystate{\myss{token := body[Token]}}
 \mystate{\myss{m':=\myangle{\mathtt{ENCLAVEMESSAGE}, \myangle{\myangle{\mathtt{Token},token},\myangle{\mathtt{Key},key}}}}}
  \mystop{\mathtt{InvokeFrom}, \mathtt{\_SELF}, m'}
  \EndIf
  \mystop{}
  \end{algorithmic}
\end{breakablealgorithm}


\end{document}
